
\newglossaryentry{odontograma}
{
    name=Odontograma,
    description={Esquema utilizado por los odontólogos que permite registrar información sobre la boca de una persona}
}

\newglossaryentry{cariología}
{
    name=Cariología,
    description={Disciplina científica dentro de la odontología que trata acerca de las interrelaciones complejas entre los fluidos orales y los depósitos bacterianos y su relación con los cambios subsecuentes en los tejidos duros dentales que provocan la caries dental}
}

\newglossaryentry{anamnesis
Cómo se pronuncia
}
{
    name=Anamnesis,
    description={Conjunto de datos que se recogen en la historia clínica de un paciente con un objetivo diagnóstico.}
}


\newglossaryentry{periodontograma}
{
    name=Periodontograma,
    description={Tabla o gráfica que muestra el estado de tus encías y el nivel de tu hueso respecto al diente}
}

\newglossaryentry{periodonto}
{
    name=Periodonto,
    description={Conjunto de estructuras que van a servir de base al diente para que éste quede fijo a su alveolo, además cumple funciones demonológicas locales y es capaz de amortiguar la carga durante la masticación.}
}



\newglossaryentry{gloss01}
{
    name=Aplicación de página única,
    description={Aplicación que carga una única página HTML y todos los componentes necesarios (tales como JavaScript y CSS) para que se ejecute la aplicación. Cualquier interacción con la página o páginas subsecuentes no requiere hacer solicitudes al servidor lo que significa que la página no es recargada.}
}

\newglossaryentry{gloss02}
{
    name=ES6,
    description={Estas siglas se refieren a las más recientes versiones del estándar de Especificación de Lenguaje ECMAScript, del cual JavaScript es una implementación. La versión ES6 (también conocida como ES2015) incluye muchas adiciones a las versiones previas tales como: funciones flecha, clases, plantillas de cadena de texto, declaraciones de variables con let y const. }
}

\newglossaryentry{gloss03}
{
    name=Compiladores,
    description={Un compilador de JavaScript toma el código JavaScript, lo transforma y devuelve en un formato diferente. El caso de uso más común es tomar código JavaScript con sintaxis ES6 y transformarlo en código que navegadores más antiguos puedan interpretar. Babel es el compilador más usado con React. }
}

\newglossaryentry{gloss04}
{
    name=Bundlers,
    description={Los bundlers toman el código JavaScript y CSS escrito como módulos separados (frecuentemente cientos de ellos), y los combina en unos cuantos archivos mejor optimizados para los navegadores. Algunos bundlers comúnmente usandos en aplicaciones de React son Webpack y Browserify. }
}

\newglossaryentry{gloss05}
{
    name=Package managers,
    description={Los package managers son herramientas que te permiten administrar las dependencias de tu proyecto. npm y Yarn son dos package managers comúnmente usados en aplicaciones de React. Ambos son clientes para el mismo registro de paquetes npm. }
}
 
 \newglossaryentry{gloss06}
{
    name=CDN,
    description={CDN son las siglas en inglés de Content Delivery Network (Red de Entrega de Contenido). Los CDN entregan contenido estático en caché desde una red de servidores alrededor del mundo. }
}

\newglossaryentry{ coetaneidad}
{
    name= Coetaneidad,
    description={Que existe al mismo tiempo que otra cosa, o que pertenece a la misma época que ella.}
}



 \newglossaryentry{gloss07}
{
    name=JSX,
    description={JSX es una extensión de sintaxis para JavaScript. Es similar a un template language, pero tiene todo el poder de JavaScript. JSX es compilado a llamadas React.createElement() que regresan simples objetos de JavaScript llamados “elementos de React”.  }
}

 \newglossaryentry{gloss08}
{
    name=Componentes,
    description={Los componentes de React son pequeños y reutilizables fragmentos de código que devuelven un elemento de React para ser renderizado en una página.  }
}

 \newglossaryentry{gloss09}
{
    name=Props,
    description={Son entradas de un componente de React. Son información que es pasada desde un componente padre a un componente hijo. Los props son de sólo lectura. No deben ser modificados de ninguna forma. Para modificar algún valor en respuesta de una entrada del usuario o una respuesta de red, usa el estado en su lugar.
  }
}


 \newglossaryentry{gloss11}
{
    name=Elementos,
    description={Los elementos de React son los bloques de construcción de una aplicación de React. Uno podría confundir los elementos con el concepto más ampliamente conocido de “componentes”. Un elemento describe lo que quieres ver en pantalla. Los elementos de React son inmutables.  }
}

 \newglossaryentry{gloss10}
{
    name=Estado,
    description={Un componente necesita estado cuando algunos datos asociados a el cambian con el tiempo. Por ejemplo, un componente Checkbox tal vez necesite isChecked en su estado, y un componente NewsFeed tal vez necesite mantener un registro de fetchedPosts en su estado. La diferencia más importante entre estado y props es que los props son pasados desde un componente padre, pero el estado es manejado por el propio componente. Un componente no puede cambiar sus props, pero puede cambiar su estado.
  }
}


 \newglossaryentry{gloss12}
{
    name=Enterprise Resource Planning,
    description={Son los sistemas de información gerenciales que integran y manejan muchos de los negocios asociados con las operaciones de producción y de los aspectos de distribución de una compañía en la producción de bienes o servicios. Entre sus módulos más comunes se encuentran el de manufactura o producción, almacenamiento, logística e información tecnológica, incluyen además la contabilidad, y suelen incluir un sistema de administración de recursos humanos, y herramientas de mercadotecnia y administración estratégica.
  }
}

 \newglossaryentry{gloss13}
{
    name=Antecedentes heredo familiares,
    description={Registro de las relaciones entre los miembros de una familia junto con sus antecedentes médicos. Esto abarca las enfermedades actuales y pasadas. En los antecedentes familiares a veces se observa la distribución de ciertas enfermedades en una familia. También se llama historia médica familiar (Instituto Nacional del Cáncer de los Institutos Nacionales de la Salud de EE. UU.). 
  }
}


\newacronym{sires}{SIRES}{Sistemas de Información de Registro Electrónico para la Salud}

\newacronym{emr}{EMR}{Electronic Medical Records }

\newacronym{ehr}{EHR}{Electronic Health Records }

\newacronym{phr}{PHR}{Personal Health Records}

\newacronym{his}{HIS}{Hspital Information System}

\newacronym{cis}{CIS}{Clinical Information System}

\newacronym{xml}{XML}{Extensible Markup Language}

\newacronym{uml}{UML}{Unified Modeling Language}

\newacronym{hms}{HMS}{Hospital Management Software}

\newacronym{lis}{LIS}{Laboratory Information System}

\newacronym{ris}{RIS}{Radiology Information System}

\newacronym{pacs}{PACS}{Picture Archiving and Communication System}

\newacronym{erp}{ERP}{Enterprise Resource Planning}
