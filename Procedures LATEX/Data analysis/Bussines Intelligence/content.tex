
%----------------------------------------------------------------------------------------
%	SECTIONS
%----------------------------------------------------------------------------------------


\section{Introducción}

SQL express
Sql Mangement studio: PAra poder crear las conulstas desde ahi



hostnanme
sqlcmd -S DESKTOP-08LNEUA -E
\section{Tabla calendairo}


\begin{lstlisting}[numbers=none]
% of Sales = DIVIDE([Asset AUM],CALCULATE([Asset AUM],ALL('Dimension')))
\end{lstlisting}
 


\begin{lstlisting}[numbers=none]
change = var _yearlast=CALCULATE(SUM([value]),FILTER(ALLEXCEPT('Table','Table'[Product]),YEAR([Date])=YEAR(MAX([Date]))-1))
var _year=CALCULATE(SUM([value]),FILTER(ALLEXCEPT('Table','Table'[Product]),YEAR([Date])=YEAR(MAX([Date]))))
var _qualast=IF(QUARTER(MAX([Date]))=1,CALCULATE(SUM([value]),FILTER(ALLEXCEPT('Table','Table'[Product]),YEAR([Date])=YEAR(MAX([Date]))-1&&QUARTER([Date])=4)),CALCULATE(SUM([value]),FILTER(ALLEXCEPT('Table','Table'[Product]),YEAR([Date])=YEAR(MAX([Date]))&&QUARTER([Date])=QUARTER(MAX([Date]))-1)))
var _quar=CALCULATE(SUM([value]),FILTER(ALLEXCEPT('Table','Table'[Product]),YEAR([Date])=YEAR(MAX([Date]))&&QUARTER([Date])=QUARTER(MAX([Date]))))
return 
IF(ISINSCOPE('Table'[Date].[Year])&&NOT(ISINSCOPE('Table'[Date].[Quarter])),DIVIDE(_year-_yearlast,_yearlast),IF(ISINSCOPE('Table'[Date].[Quarter]),DIVIDE(_quar-_qualast,_qualast)))
\end{lstlisting}



SELECT name AS [Login Name], type_desc AS [Account Type]
FROM sys.server_principals 
WHERE TYPE IN ('U', 'S', 'G')
and name not like '%##%'
ORDER BY name, type_desc

GO

sqlcmd -S DESKTOP-08LNEUA -U osvaldohm





let fnDateTable = (StartDate as date, EndDate as date, FYStartMonth as number) as table =>
  let
    DayCount = Duration.Days(Duration.From(EndDate - StartDate)),
    Source = List.Dates(StartDate,DayCount,#duration(1,0,0,0)),
    TableFromList = Table.FromList(Source, Splitter.SplitByNothing()),   
    ChangedType = Table.TransformColumnTypes(TableFromList,{{"Column1", type date}}),
    RenamedColumns = Table.RenameColumns(ChangedType,{{"Column1", "Date"}}),
    InsertYear = Table.AddColumn(RenamedColumns, "Year", each Date.Year([Date]),type text),
    InsertYearNumber = Table.AddColumn(RenamedColumns, "YearNumber", each Date.Year([Date])),
    InsertQuarter = Table.AddColumn(InsertYear, "QuarterOfYear", each Date.QuarterOfYear([Date])),
    InsertMonth = Table.AddColumn(InsertQuarter, "MonthOfYear", each Date.Month([Date]), type text),
    InsertDay = Table.AddColumn(InsertMonth, "DayOfMonth", each Date.Day([Date])),
    InsertDayInt = Table.AddColumn(InsertDay, "DateInt", each [Year] * 10000 + [MonthOfYear] * 100 + [DayOfMonth]),
    InsertMonthName = Table.AddColumn(InsertDayInt, "MonthName", each Date.ToText([Date], "MMMM"), type text),
    InsertCalendarMonth = Table.AddColumn(InsertMonthName, "MonthInCalendar", each (try(Text.Range([MonthName],0,3)) otherwise [MonthName]) & " " & Number.ToText([Year])),
    InsertCalendarQtr = Table.AddColumn(InsertCalendarMonth, "QuarterInCalendar", each "Q" & Number.ToText([QuarterOfYear]) & " " & Number.ToText([Year])),
    InsertDayWeek = Table.AddColumn(InsertCalendarQtr, "DayInWeek", each Date.DayOfWeek([Date])),
    InsertDayName = Table.AddColumn(InsertDayWeek, "DayOfWeekName", each Date.ToText([Date], "dddd"), type text),
    InsertWeekEnding = Table.AddColumn(InsertDayName, "WeekEnding", each Date.EndOfWeek([Date]), type date),
    InsertWeekNumber= Table.AddColumn(InsertWeekEnding, "Week Number", each Date.WeekOfYear([Date])),
    InsertMonthnYear = Table.AddColumn(InsertWeekNumber,"MonthnYear", each [Year] * 10000 + [MonthOfYear] * 100),
    InsertQuarternYear = Table.AddColumn(InsertMonthnYear,"QuarternYear", each [Year] * 10000 + [QuarterOfYear] * 100),
    ChangedType1 = Table.TransformColumnTypes(InsertQuarternYear,{{"QuarternYear", Int64.Type},{"Week Number", Int64.Type},{"Year", type text},{"MonthnYear", Int64.Type}, {"DateInt", Int64.Type}, {"DayOfMonth", Int64.Type}, {"MonthOfYear", Int64.Type}, {"QuarterOfYear", Int64.Type}, {"MonthInCalendar", type text}, {"QuarterInCalendar", type text}, {"DayInWeek", Int64.Type}}),
    InsertShortYear = Table.AddColumn(ChangedType1, "ShortYear", each Text.End(Text.From([Year]), 2), type text),
    AddFY = Table.AddColumn(InsertShortYear, "FY", each "FY"&(if [MonthOfYear]>=FYStartMonth then Text.From(Number.From([ShortYear])+1) else [ShortYear]))
in
    AddFY
in
    fnDateTable


let
    // Start FYMonth is 11 November
    Source = #"DATES QUERY"(#date(2014, 1, 1), #date(2025, 01, 01), 11),
    #"Renamed Columns" = Table.RenameColumns(Source,{{"MonthName", "Month"}, {"MonthInCalendar", "Month & Year"}})
in
    #"Renamed Columns"




All databses
SELECT name FROM master.sys.databases
GO
USE MundoBinario
GO


	Run cmd as asmind

	sqlcmd -S DESKTOP-08LNEUA\SQLEXPRESS -E

sentencia  SQL
GO 





Mostrar un plan de ejecución real
Se aplica a: SQL Server (todas las versiones compatibles)  Azure SQL Managed Instance Database  Azure SQL
En este tema se describe cómo generar planes de ejecución gráficos reales mediante SQL Server Management Studio. Los planes de ejecución reales se generan después de que se ejecuten las consultas o lotes de Transact-SQL. Por este motivo, un plan de ejecución real contiene información de tiempo de ejecución, como métricas de uso real de recursos o advertencias en tiempo de ejecución (si las hubiera). El plan de ejecución que se genera muestra el plan de ejecución de consultas real que el motor de base de datos de SQL Server usó para ejecutar las consultas.

Para usar esta característica, los usuarios deben tener los permisos adecuados para ejecutar las consultas de Transact-SQL para las que se genera un plan de ejecución gráfico y se les debe conceder el permiso SHOWPLAN para todas las bases de datos a las que hace referencia la consulta..


Inclue Actual Execution Plan Button
Includwe Live Stastus 





Hiearchy = 
var isyearinscope = ISINSCOPE(Date_Master[Date].[Year])
var isquarterinscope = ISINSCOPE(Date_Master[Date].[Quarter])
var ismonthinscope = ISINSCOPE(Date_Master[Date].[Month])
var isdayinscope = ISINSCOPE(Date_Master[Date].[Day])
return
IF(
    isdayinscope,"D",
    IF(
        ismonthinscope,"M",
        IF(
            isquarterinscope,"Q",
            IF(
                isyearinscope,"Y",
                "Y"
                )
            )
        )
    )



    var TotalMonth = CALCULATE(SUM(financials[ Sales]), FILTER(ALLSELECTED('Calendar'),'Calendar'[Date] <= ENDOFMONTH('Calendar'[Date]) && 'Calendar'[Date] >= STARTOFMONTH('Calendar'[Date])))



\begin{lstlisting}[numbers=none]
% Percentage = 

var Total = CALCULATE(SUM(financials[ Sales]), ALLSELECTED('Calendar'))

var CurrentLastDate = MAX('Calendar'[Date])

var TotalMonth = CALCULATE(SUM(financials[ Sales]), FILTER(ALLSELECTED('Calendar'), 'Calendar'[Date] >= DATE(YEAR(CurrentLastDate),MONTH(CurrentLastDate),1) && 'Calendar'[Date] <= EOMONTH(CurrentLastDate,0)))

var TotalYear = CALCULATE(SUM(financials[ Sales]), FILTER(ALLSELECTED('Calendar'), 'Calendar'[Date] >= DATE(YEAR(CurrentLastDate),1,1) && 'Calendar'[Date] <=  DATE(YEAR(CurrentLastDate),12,31)))

var TotalQuarter = CALCULATE(SUM(financials[ Sales]), FILTER(ALLSELECTED('Calendar'), 'Calendar'[Date] >= DATE(YEAR(CurrentLastDate),INT(FORMAT(CurrentLastDate,"q"))*3-2,1) && 'Calendar'[Date] <=  DATE(YEAR(CurrentLastDate),INT(FORMAT(CurrentLastDate,"q"))*3,EOMONTH(DATE(YEAR(CurrentLastDate),INT(FORMAT(CurrentLastDate,"q"))*3,1),0))))


var Sales = SUM(financials[ Sales])

var Denominator  = SWITCH( true(), ISINSCOPE('Calendar'[Date].[Day]),TotalMonth,ISINSCOPE('Calendar'[Date].[Month]),TotalQuarter,ISINSCOPE('Calendar'[Date].[Quarter]),TotalYear,ISINSCOPE('Calendar'[Date].[Year]),Total)

var Percentage = DIVIDE(Sales, Denominator,1)

return Percentage
\end{lstlisting}



CREATE DATABASE PRUEBAT

USE PRUEBAT 

CREATE TABLE Dim_sucursal (
    SucursalKey integer IDENTITY(1,1) PRIMARY KEY,
	Nombre varchar(255),
	Ubicacion varchar(255),
	Region varchar(255),
	Canal varchar(255)
);

CREATE TABLE Dim_articulo (
    ArticuloKey integer IDENTITY(1,1) PRIMARY KEY,
	Descripcion varchar(255),
	Proveedor_Id varchar(255),
	Fecha_alta date	
);

CREATE TABLE Dim_tiempo (
    TiempoKey integer IDENTITY(1,1) PRIMARY KEY,
	Fecha date,
	Anio integer,
	Mes integer,
	Hora time
);


CREATE TABLE Hecho_transacciones (
    TransactionID integer IDENTITY(1,1) PRIMARY KEY,
	SKU varchar(255),
	Monto_venta money,
	Promocion decimal,
	Cantidad integer,
	Impuesto decimal(4,3),
	Descuento decimal(4,3),
	TiempoKey integer FOREIGN KEY REFERENCES Dim_tiempo(TiempoKey),
	SucursalKey integer FOREIGN KEY REFERENCES Dim_sucursal(SucursalKey),
	ArticuloKey integer FOREIGN KEY REFERENCES Dim_articulo(ArticuloKey)
);

INSERT INTO Dim_sucursal (Nombre,Ubicacion,Region,Canal) VALUES ('Sucursal Norte','Medellin','LATAM','A');
INSERT INTO Dim_sucursal (Nombre,Ubicacion,Region,Canal) VALUES ('Sucursal Sur','México','LATAM','B');
INSERT INTO Dim_sucursal (Nombre,Ubicacion,Region,Canal) VALUES ('Sucursal Este','Argentina','LATAM','C');
INSERT INTO Dim_sucursal (Nombre,Ubicacion,Region,Canal) VALUES ('Sucursal Oeste','Perú','LATAM','D');
INSERT INTO Dim_sucursal (Nombre,Ubicacion,Region,Canal) VALUES ('Sucursal Centro','Guatemala','LATAM','E');

SELECT * FROM  Dim_sucursal

INSERT INTO Dim_articulo (Descripcion,Proveedor_Id,Fecha_alta) VALUES ('Laptop','123','20120618 10:34:09 AM');
INSERT INTO Dim_articulo (Descripcion,Proveedor_Id,Fecha_alta) VALUES ('PC','123','20120618 11:34:09 AM');
INSERT INTO Dim_articulo (Descripcion,Proveedor_Id,Fecha_alta) VALUES ('Celular','123','20120618 12:34:09 AM');
INSERT INTO Dim_articulo (Descripcion,Proveedor_Id,Fecha_alta) VALUES ('Reloj','123','20120618 06:34:09 PM');
INSERT INTO Dim_articulo (Descripcion,Proveedor_Id,Fecha_alta) VALUES ('Bocina','123','20120618 08:34:09 PM');
INSERT INTO Dim_articulo (Descripcion,Proveedor_Id,Fecha_alta) VALUES ('Lampara','123','20120618 08:34:09 PM');

SELECT * FROM  Dim_articulo

\begin{lstlisting}[numbers=none]
Declare @numSucursales int
Set @numSucursales = ( Select Count(*) From Dim_sucursal )

DECLARE @StartDate AS date;
DECLARE @EndDate AS date;
SELECT @StartDate = '01/01/2019', -- Date Format - DD/MM/YYY
       @EndDate   = '12/31/2022';
Declare @SalesDate datetime;
Set @SalesDate =  DATEADD(DAY, RAND(CHECKSUM(NEWID()))*(1+DATEDIFF(DAY, @StartDate, @EndDate)),@StartDate);

INSERT INTO Dim_tiempo (Fecha,Anio,Mes,Hora) VALUES (@SalesDate,year( @SalesDate ),month( @SalesDate ),CONVERT(VARCHAR(8),@SalesDate,108));
INSERT INTO Hecho_transacciones (ArticuloKey,Monto_venta,Promocion,Cantidad,Impuesto,Descuento,TiempoKey,SucursalKey) VALUES (1,RAND()*(5000-100)+100,3,RAND()*(10-1)+1,0.16,0.2,SCOPE_IDENTITY(),RAND()*(@numSucursales-1)+1);
\end{lstlisting}

SELECT * FROM  Hecho_transacciones INNER JOIN Dim_tiempo ON Hecho_transacciones.TiempoKey=Dim_tiempo.TiempoKey

# Canal
SELECT Canal, SUM(Monto_venta) Ventas_Canal FROM Hecho_transacciones LEFT JOIN Dim_sucursal ON Hecho_transacciones.SucursalKey=Dim_sucursal.SucursalKey GROUP BY Dim_sucursal.Canal;
# Por año
SELECT SUM(Monto_venta) Ventas_Anio FROM Hecho_transacciones LEFT JOIN Dim_tiempo ON Hecho_transacciones.TiempoKey=Dim_tiempo.TiempoKey GROUP BY Anio;
# Por mes
SELECT SUM(Monto_venta) Ventas_Mes,Mes FROM Hecho_transacciones LEFT JOIN Dim_tiempo ON Hecho_transacciones.TiempoKey=Dim_tiempo.TiempoKey GROUP BY Mes;
# Ultima fechad de venta
SELECT max_date = max( Fecha ) FROM Hecho_transacciones INNER JOIN Dim_tiempo ON Hecho_transacciones.TiempoKey=Dim_tiempo.TiempoKey;


DECLARE  @columns NVARCHAR(MAX) = '';
SELECT @columns += QUOTENAME(Anio) + ',' FROM Dim_tiempo GROUP BY Anio;
SET @columns = LEFT(@columns, LEN(@columns) - 1);
PRINT @columns;

SELECT Canal, SUM(Monto_venta) Ventas_Canal FROM Hecho_transacciones LEFT JOIN Dim_sucursal ON Hecho_transacciones.SucursalKey=Dim_sucursal.SucursalKey INNER JOIN Dim_tiempo ON Hecho_transacciones.TiempoKey=Dim_tiempo.TiempoKey  GROUP BY Dim_sucursal.Canal;

# No sold producto
select Descripcion from Dim_articulo left join Hecho_transacciones on Dim_articulo.ArticuloKey = Hecho_transacciones.ArticuloKey where Hecho_transacciones.ArticuloKey is null;

# Ranking ventas 
Select Nombre, RANK () OVER ( PARTITION BY Nombre ORDER BY Monto_venta DESC ) price_rank From Dim_sucursal INNER JOIN Hecho_transacciones ON Dim_sucursal.SucursalKey = Hecho_transacciones.SucursalKey 


# Stored proocuder
CREATE PROCEDURE  Saludar AS PRINT 'Hola, Como estas?';
EXECUTE Saludar;

create procedure InsertSale
@Monto_venta money,
@Promocion decimal,
@Cantidad integer,
@Impuesto decimal(4,3),
@Descuento decimal(4,3),
@TiempoKey integer,
@SucursalKey integer,
@ArticuloKey integer
as
Begin
INSERT INTO Hecho_transacciones (ArticuloKey,Monto_venta,Promocion,Cantidad,Impuesto,Descuento,TiempoKey,SucursalKey) VALUES (@ArticuloKey,@Monto_venta,@Promocion,@Cantidad,@Impuesto,@Descuento,@TiempoKey,@SucursalKey);
END

Uno esx le modelo realacional de tabklas atrivutos

Modelo entidad-relacion

Rombo es relacion


La diferencia básica entre el Modelo E-R y el Modelo Relacional es que el modelo E-R trata específicamente con las entidades y sus relaciones. Por otro lado, el Modelo Relacional se ocupa de las Tablas y de la relación entre los datos de esas tablas.




Could not obtain information about Windows NT group/user
Change the owner to sa. Here are the steps I took to solve this issue:

Right-Click on the database and select properties

Click on Files under the Select a page

Under the Owner, but just below the Database Name on the right-hand pane, select sa as the owner.


ETL Diagram: Visio or Excel :-)

Remember for ETL diagrams you need 3 type:

PDD - Process Dependency Diagram: Shows the sequence of execution and dependency for a bunch of ETL processes.
DFD - Data Flow Diagram: Show how the data is flowing from one table or file to another. (You can even create data mapping document in excel)
ETL Design: Shows how the data is being processing by a single ETL process. ex. read, filter, lookup, join, aggregate and load.


Stagin database
Datawarehouse


ODS 

OLTP sistema transaccional
OLAP sistema de analisis

Ralph kimbal


Dimension lentametne cambiante

Tipo 1 sobbrescibre como cambiar el numero telefonico
Tipo 2 van agregando por ejemplo campo current, registro de la  hitria de domicilios
Tipo 3 no es un nuevo registro, es una acualización porque tiene una columna 

\begin{lstlisting}[numbers=none]
    % of Sales = DIVIDE([Asset AUM],CALCULATE([Asset AUM],ALL('Dimension')))
    \end{lstlisting}
     
    






    path item
path index


PATH 
Seguridad en filas

FLujo de datos con cargas incrementales

Actividadades
Limpiar 
Cerrar
Actualizar registros 


Una tabla a la tabla de hechos.

